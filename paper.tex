%% This style is provided for the ICSE 2015 main conference,
%% ICSE 2015 co-located events, and ICSE 2015 workshops.

%% bare_conf_ICSE15.tex
%% V1.4
%% 2014/05/22


%% This is a skeleton file demonstrating the use of IEEEtran.cls
%% (requires IEEEtran.cls version 1.7 or later) with an IEEE conference paper.
%%
%% Support sites:
%% http://www.michaelshell.org/tex/ieeetran/
%% http://www.ctan.org/tex-archive/macros/latex/contrib/IEEEtran/
%% and
%% http://www.ieee.org/

%%*************************************************************************
%% Legal Notice:
%% This code is offered as-is without any warranty either expressed or
%% implied; without even the implied warranty of MERCHANTABILITY or
%% FITNESS FOR A PARTICULAR PURPOSE!
%% User assumes all risk.
%% In no event shall IEEE or any contributor to this code be liable for
%% any damages or losses, including, but not limited to, incidental,
%% consequential, or any other damages, resulting from the use or misuse
%% of any information contained here.
%%
%% All comments are the opinions of their respective authors and are not
%% necessarily endorsed by the IEEE.
%%
%% This work is distributed under the LaTeX Project Public License (LPPL)
%% ( http://www.latex-project.org/ ) version 1.3, and may be freely used,
%% distributed and modified. A copy of the LPPL, version 1.3, is included
%% in the base LaTeX documentation of all distributions of LaTeX released
%% 2003/12/01 or later.
%% Retain all contribution notices and credits.
%% ** Modified files should be clearly indicated as such, including  **
%% ** renaming them and changing author support contact information. **
%%
%% File list of work: IEEEtran.cls, IEEEtran_HOWTO.pdf, bare_adv.tex,
%%                    bare_conf.tex, bare_jrnl.tex, bare_jrnl_compsoc.tex
%%*************************************************************************

% *** Authors should verify (and, if needed, correct) their LaTeX system  ***
% *** with the testflow diagnostic prior to trusting their LaTeX platform ***
% *** with production work. IEEE's font choices can trigger bugs that do  ***
% *** not appear when using other class files.                            ***
% The testflow support page is at:
% http://www.michaelshell.org/tex/testflow/



% Note that the a4paper option is mainly intended so that authors in
% countries using A4 can easily print to A4 and see how their papers will
% look in print - the typesetting of the document will not typically be
% affected with changes in paper size (but the bottom and side margins will).
% Use the testflow package mentioned above to verify correct handling of
% both paper sizes by the user's LaTeX system.
%
% Also note that the "draftcls" or "draftclsnofoot", not "draft", option
% should be used if it is desired that the figures are to be displayed in
% draft mode.
%
\documentclass[conference]{IEEEtran}
%
% If IEEEtran.cls has not been installed into the LaTeX system files,
% manually specify the path to it like:
% \documentclass[conference]{../sty/IEEEtran}


\usepackage{graphicx}


% Some very useful LaTeX packages include:
% (uncomment the ones you want to load)


% *** MISC UTILITY PACKAGES ***
%
%\usepackage{ifpdf}
% Heiko Oberdiek's ifpdf.sty is very useful if you need conditional
% compilation based on whether the output is pdf or dvi.
% usage:
% \ifpdf
%   % pdf code
% \else
%   % dvi code
% \fi
% The latest version of ifpdf.sty can be obtained from:
% http://www.ctan.org/tex-archive/macros/latex/contrib/oberdiek/
% Also, note that IEEEtran.cls V1.7 and later provides a builtin
% \ifCLASSINFOpdf conditional that works the same way.
% When switching from latex to pdflatex and vice-versa, the compiler may
% have to be run twice to clear warning/error messages.






% *** CITATION PACKAGES ***
%
%\usepackage{cite}
% cite.sty was written by Donald Arseneau
% V1.6 and later of IEEEtran pre-defines the format of the cite.sty package
% \cite{} output to follow that of IEEE. Loading the cite package will
% result in citation numbers being automatically sorted and properly
% "compressed/ranged". e.g., [1], [9], [2], [7], [5], [6] without using
% cite.sty will become [1], [2], [5]--[7], [9] using cite.sty. cite.sty's
% \cite will automatically add leading space, if needed. Use cite.sty's
% noadjust option (cite.sty V3.8 and later) if you want to turn this off.
% cite.sty is already installed on most LaTeX systems. Be sure and use
% version 4.0 (2003-05-27) and later if using hyperref.sty. cite.sty does
% not currently provide for hyperlinked citations.
% The latest version can be obtained at:
% http://www.ctan.org/tex-archive/macros/latex/contrib/cite/
% The documentation is contained in the cite.sty file itself.






% *** GRAPHICS RELATED PACKAGES ***
%
\ifCLASSINFOpdf
  % \usepackage[pdftex]{graphicx}
  % declare the path(s) where your graphic files are
  % \graphicspath{{../pdf/}{../jpeg/}}
  % and their extensions so you won't have to specify these with
  % every instance of \includegraphics
  % \DeclareGraphicsExtensions{.pdf,.jpeg,.png}
\else
  % or other class option (dvipsone, dvipdf, if not using dvips). graphicx
  % will default to the driver specified in the system graphics.cfg if no
  % driver is specified.
  % \usepackage[dvips]{graphicx}
  % declare the path(s) where your graphic files are
  % \graphicspath{{../eps/}}
  % and their extensions so you won't have to specify these with
  % every instance of \includegraphics
  % \DeclareGraphicsExtensions{.eps}
\fi
% graphicx was written by David Carlisle and Sebastian Rahtz. It is
% required if you want graphics, photos, etc. graphicx.sty is already
% installed on most LaTeX systems. The latest version and documentation can
% be obtained at:
% http://www.ctan.org/tex-archive/macros/latex/required/graphics/
% Another good source of documentation is "Using Imported Graphics in
% LaTeX2e" by Keith Reckdahl which can be found as epslatex.ps or
% epslatex.pdf at: http://www.ctan.org/tex-archive/info/
%
% latex, and pdflatex in dvi mode, support graphics in encapsulated
% postscript (.eps) format. pdflatex in pdf mode supports graphics
% in .pdf, .jpeg, .png and .mps (metapost) formats. Users should ensure
% that all non-photo figures use a vector format (.eps, .pdf, .mps) and
% not a bitmapped formats (.jpeg, .png). IEEE frowns on bitmapped formats
% which can result in "jaggedy"/blurry rendering of lines and letters as
% well as large increases in file sizes.
%
% You can find documentation about the pdfTeX application at:
% http://www.tug.org/applications/pdftex





% *** MATH PACKAGES ***
%
%\usepackage[cmex10]{amsmath}
% A popular package from the American Mathematical Society that provides
% many useful and powerful commands for dealing with mathematics. If using
% it, be sure to load this package with the cmex10 option to ensure that
% only type 1 fonts will utilized at all point sizes. Without this option,
% it is possible that some math symbols, particularly those within
% footnotes, will be rendered in bitmap form which will result in a
% document that can not be IEEE Xplore compliant!
%
% Also, note that the amsmath package sets \interdisplaylinepenalty to 10000
% thus preventing page breaks from occurring within multiline equations. Use:
%\interdisplaylinepenalty=2500
% after loading amsmath to restore such page breaks as IEEEtran.cls normally
% does. amsmath.sty is already installed on most LaTeX systems. The latest
% version and documentation can be obtained at:
% http://www.ctan.org/tex-archive/macros/latex/required/amslatex/math/





% *** SPECIALIZED LIST PACKAGES ***
%
%\usepackage{algorithmic}
% algorithmic.sty was written by Peter Williams and Rogerio Brito.
% This package provides an algorithmic environment fo describing algorithms.
% You can use the algorithmic environment in-text or within a figure
% environment to provide for a floating algorithm. Do NOT use the algorithm
% floating environment provided by algorithm.sty (by the same authors) or
% algorithm2e.sty (by Christophe Fiorio) as IEEE does not use dedicated
% algorithm float types and packages that provide these will not provide
% correct IEEE style captions. The latest version and documentation of
% algorithmic.sty can be obtained at:
% http://www.ctan.org/tex-archive/macros/latex/contrib/algorithms/
% There is also a support site at:
% http://algorithms.berlios.de/index.html
% Also of interest may be the (relatively newer and more customizable)
% algorithmicx.sty package by Szasz Janos:
% http://www.ctan.org/tex-archive/macros/latex/contrib/algorithmicx/




% *** ALIGNMENT PACKAGES ***
%
%\usepackage{array}
% Frank Mittelbach's and David Carlisle's array.sty patches and improves
% the standard LaTeX2e array and tabular environments to provide better
% appearance and additional user controls. As the default LaTeX2e table
% generation code is lacking to the point of almost being broken with
% respect to the quality of the end results, all users are strongly
% advised to use an enhanced (at the very least that provided by array.sty)
% set of table tools. array.sty is already installed on most systems. The
% latest version and documentation can be obtained at:
% http://www.ctan.org/tex-archive/macros/latex/required/tools/


%\usepackage{mdwmath}
%\usepackage{mdwtab}
% Also highly recommended is Mark Wooding's extremely powerful MDW tools,
% especially mdwmath.sty and mdwtab.sty which are used to format equations
% and tables, respectively. The MDWtools set is already installed on most
% LaTeX systems. The lastest version and documentation is available at:
% http://www.ctan.org/tex-archive/macros/latex/contrib/mdwtools/


% IEEEtran contains the IEEEeqnarray family of commands that can be used to
% generate multiline equations as well as matrices, tables, etc., of high
% quality.


%\usepackage{eqparbox}
% Also of notable interest is Scott Pakin's eqparbox package for creating
% (automatically sized) equal width boxes - aka "natural width parboxes".
% Available at:
% http://www.ctan.org/tex-archive/macros/latex/contrib/eqparbox/





% *** SUBFIGURE PACKAGES ***
%\usepackage[tight,footnotesize]{subfigure}
% subfigure.sty was written by Steven Douglas Cochran. This package makes it
% easy to put subfigures in your figures. e.g., "Figure 1a and 1b". For IEEE
% work, it is a good idea to load it with the tight package option to reduce
% the amount of white space around the subfigures. subfigure.sty is already
% installed on most LaTeX systems. The latest version and documentation can
% be obtained at:
% http://www.ctan.org/tex-archive/obsolete/macros/latex/contrib/subfigure/
% subfigure.sty has been superceeded by subfig.sty.



%\usepackage[caption=false]{caption}
%\usepackage[font=footnotesize]{subfig}
% subfig.sty, also written by Steven Douglas Cochran, is the modern
% replacement for subfigure.sty. However, subfig.sty requires and
% automatically loads Axel Sommerfeldt's caption.sty which will override
% IEEEtran.cls handling of captions and this will result in nonIEEE style
% figure/table captions. To prevent this problem, be sure and preload
% caption.sty with its "caption=false" package option. This is will preserve
% IEEEtran.cls handing of captions. Version 1.3 (2005/06/28) and later
% (recommended due to many improvements over 1.2) of subfig.sty supports
% the caption=false option directly:
%\usepackage[caption=false,font=footnotesize]{subfig}
%
% The latest version and documentation can be obtained at:
% http://www.ctan.org/tex-archive/macros/latex/contrib/subfig/
% The latest version and documentation of caption.sty can be obtained at:
% http://www.ctan.org/tex-archive/macros/latex/contrib/caption/




% *** FLOAT PACKAGES ***
%
%\usepackage{fixltx2e}
% fixltx2e, the successor to the earlier fix2col.sty, was written by
% Frank Mittelbach and David Carlisle. This package corrects a few problems
% in the LaTeX2e kernel, the most notable of which is that in current
% LaTeX2e releases, the ordering of single and double column floats is not
% guaranteed to be preserved. Thus, an unpatched LaTeX2e can allow a
% single column figure to be placed prior to an earlier double column
% figure. The latest version and documentation can be found at:
% http://www.ctan.org/tex-archive/macros/latex/base/



%\usepackage{stfloats}
% stfloats.sty was written by Sigitas Tolusis. This package gives LaTeX2e
% the ability to do double column floats at the bottom of the page as well
% as the top. (e.g., "\begin{figure*}[!b]" is not normally possible in
% LaTeX2e). It also provides a command:
%\fnbelowfloat
% to enable the placement of footnotes below bottom floats (the standard
% LaTeX2e kernel puts them above bottom floats). This is an invasive package
% which rewrites many portions of the LaTeX2e float routines. It may not work
% with other packages that modify the LaTeX2e float routines. The latest
% version and documentation can be obtained at:
% http://www.ctan.org/tex-archive/macros/latex/contrib/sttools/
% Documentation is contained in the stfloats.sty comments as well as in the
% presfull.pdf file. Do not use the stfloats baselinefloat ability as IEEE
% does not allow \baselineskip to stretch. Authors submitting work to the
% IEEE should note that IEEE rarely uses double column equations and
% that authors should try to avoid such use. Do not be tempted to use the
% cuted.sty or midfloat.sty packages (also by Sigitas Tolusis) as IEEE does
% not format its papers in such ways.





% *** PDF, URL AND HYPERLINK PACKAGES ***
%
%\usepackage{url}
% url.sty was written by Donald Arseneau. It provides better support for
% handling and breaking URLs. url.sty is already installed on most LaTeX
% systems. The latest version can be obtained at:
% http://www.ctan.org/tex-archive/macros/latex/contrib/misc/
% Read the url.sty source comments for usage information. Basically,
% \url{my_url_here}.





% *** Do not adjust lengths that control margins, column widths, etc. ***
% *** Do not use packages that alter fonts (such as pslatex).         ***
% There should be no need to do such things with IEEEtran.cls V1.6 and later.
% (Unless specifically asked to do so by the journal or conference you plan
% to submit to, of course. )


% correct bad hyphenation here
\hyphenation{op-tical net-works semi-conduc-tor}

\usepackage{listings}

\lstset{
	numbers=left,
	numberstyle=\tiny,
	language={Java},
	mathescape=true,
	flexiblecolumns=true,
	morekeywords={def,Int,call,method,var,assert,share,unshare,acquire,release,fork,join,free,invariant,requires,ensures,acc,rd,old},
	basicstyle=\sffamily\small,
	moredelim=[is][\itshape]{@}{@},
	stepnumber=1,
	numbersep=2pt} 

\begin{document}
%
% paper title
% can use linebreaks \\ within to get better formatting as desired
\title{Recipe: Relaxed Sound Predictive Analysis}


% author names and affiliations
% use a multiple column layout for up to three different
% affiliations
%\author{\IEEEauthorblockN{Michael Shell}
%\IEEEauthorblockA{School of Electrical and\\Computer Engineering\\
%Georgia Institute of Technology\\
%Atlanta, Georgia 30332--0250\\
%Email: http://www.michaelshell.org/contact.html}
%\and
%\IEEEauthorblockN{Homer Simpson}
%\IEEEauthorblockA{Twentieth Century Fox\\
%Springfield, USA\\
%Email: homer@thesimpsons.com}
%\and
%\IEEEauthorblockN{James Kirk\\ and Montgomery Scott}
%\IEEEauthorblockA{Starfleet Academy\\
%San Francisco, California 96678-2391\\
%Telephone: (800) 555--1212\\
%Fax: (888) 555--1212}}

% conference papers do not typically use \thanks and this command
% is locked out in conference mode. If really needed, such as for
% the acknowledgment of grants, issue a \IEEEoverridecommandlockouts
% after \documentclass

% for over three affiliations, or if they all won't fit within the width
% of the page, use this alternative format:
%
%\author{\IEEEauthorblockN{Michael Shell\IEEEauthorrefmark{1},
%Homer Simpson\IEEEauthorrefmark{2},
%James Kirk\IEEEauthorrefmark{3},
%Montgomery Scott\IEEEauthorrefmark{3} and
%Eldon Tyrell\IEEEauthorrefmark{4}}
%\IEEEauthorblockA{\IEEEauthorrefmark{1}School of Electrical and Computer Engineering\\
%Georgia Institute of Technology,
%Atlanta, Georgia 30332--0250\\ Email: see http://www.michaelshell.org/contact.html}
%\IEEEauthorblockA{\IEEEauthorrefmark{2}Twentieth Century Fox, Springfield, USA\\
%Email: homer@thesimpsons.com}
%\IEEEauthorblockA{\IEEEauthorrefmark{3}Starfleet Academy, San Francisco, California 96678-2391\\
%Telephone: (800) 555--1212, Fax: (888) 555--1212}
%\IEEEauthorblockA{\IEEEauthorrefmark{4}Tyrell Inc., 123 Replicant Street, Los Angeles, California 90210--4321}}




% use for special paper notices
%\IEEEspecialpapernotice{(Invited Paper)}




% make the title area
\maketitle


\begin{abstract}
%\boldmath
The abstract goes here.
\end{abstract}
% IEEEtran.cls defaults to using nonbold math in the Abstract.
% This preserves the distinction between vectors and scalars. However,
% if the conference you are submitting to favors bold math in the abstract,
% then you can use LaTeX's standard command \boldmath at the very start
% of the abstract to achieve this. Many IEEE journals/conferences frown on
% math in the abstract anyway.

% no keywords




% For peer review papers, you can put extra information on the cover
% page as needed:
% \ifCLASSOPTIONpeerreview
% \begin{center} \bfseries EDICS Category: 3-BBND \end{center}
% \fi
%
% For peerreview papers, this IEEEtran command inserts a page break and
% creates the second title. It will be ignored for other modes.
\IEEEpeerreviewmaketitle



%Limitations of static analysis: scalability when it is path sensitive, difficulty in modeling heap, cannot reason about concrete values, have to use
%abstraction, so that may not be sound, dynamic analysis is precise, easily scale to long and complex runs, but it only reasons about problems that
%did occur. 

%Predictive analysis is so good

%But it restricts too much, only relax schedule space and a little bit of the dependence space.

%We observe many dynamic aspects can be relaxed without affecting soundness.  as a result, instead of reasoning one
%execution, we can reason the neighborhood of the execution, in order words, we leverage dynamic analysis to provide
%the information that is difficult to analyze statically such as aliasing and a long execution path that covers the
%intended workflow, then we relax the predictive analysis to analyze neighborhood.  


\section{Introduction}\label{Se:introduction}
Program analyses can be largely classified to static and dynamic analyses. Static analyses abstract programs statically.
They are usually complete and do not require concrete program inputs to drive the analyses. However, they have difficulty in
scaling to large and complex programs when the analyses are path-sensitive, due to the sheer number of paths and the 
complexity of the paths. Path-insensitive analyses often do not have sufficient precision on the other hand. 
Static analyses also have difficulty in modeling complex aliasing behavior. In contrast, dynamic analysese focus 
on analyzing a few executions, usually just one. They are sound and easily scale to long and complex executions. But they 
can only reason about properties that manifest themselves in the execution(s). 

Recently, a novel kind of analysis called predictive analysis~\cite{...} was proposed. It has the capabilities of 
achieving soundness as dynamic analyses and reasoning about properties that did not occur during execution 
just like in static analysis. The basic idea is to leverage dynamic analysis to provide the information that 
is difficult to acquire through static analysis, such as a complex execution path that covers important functionalities 
and aliasing information. Note that during execution, we know precisely which write access reaches any given read 
access. While such information is encoded in a trace, predictive analysis reaons about mutation of the trace to 
expose problems or study properties that may not manifest during execution. It was used in sound data race detection. 
In particular, it collects a trace of threaded execution that contains rich runtime information such as concrete values, 
execution path, and memory accsses. It then leverages constraint solving to reason about if races can occur with
a different thread schedule, {\em while enforcing the critical runtime information (e.g. values, thread
local paths, and memory accesses) remains the same during schedule pertubation}. The rationale is that 
it becomes as difficult as static analysis if we allow these critical runtime information change. 
Preditive analysis has been shown to be very effective. It can detect real data races from complex programs.
And more importantly, it guarantees the results are sound (i.e. no false positives).

Despite its effectiveness, we observe that existing predictive analyes are too restrictive. They 
require too much runtime information to remain unchanged during analysis such that coverage is unnecessarily
limited. 


\begin{figure}
\centering
\begin{tabular}{ll}
\multicolumn{2}{c}{{\tt x = 0; y = 0;}} \\  % z = 3;
%\multicolumn{2}{c}{{\color{red} {\tt z = 2;}}} \\
\hline
\multicolumn{1}{c}{$T_1$} & \multicolumn{1}{c}{$T_2$} \\
\hline
{\tt 1: y = 3;} & \\
{\tt 2: x = 1;} & \\
{\tt 3: y = 5;} & \\
& {\tt 4: if (y > 2)} \\
& {\tt 5:~~z=1/x;} \\	
& {\color{red} {\tt 6: else}} \\
& {\color{red} {\tt 7:~~w=2/x;}}
\end{tabular}
\caption{Example illustrating ordering constraints beyond synchronization primitives}
\label{fig:running}
\end{figure}

Consider the example in Figure \ref{fig:running}, which shows an execution trace of two threads. The statements in red 
denote an unexplored branch outside the trace. Existing predictive analyses are unable to detect the race between 
lines {\tt 2} and {\tt 5} because they require that during trace permutations (by different schedules), a variable must not have
a different value in a permuted trace\footnote{They allow the variable involved in a race to have different values.}.
As such, they cannot execute line {\tt 4} before line {\tt 3}, which is needed to expose the race between lines {\tt 2}
and {\tt 5}. 
%the dependence between lines {\tt 3} and {\tt 4}, which is a barrier to the needed reorderings. 
A second race between lines {\tt 2} and {\tt 7} is also missed as the mutated trace must have the same thread local
path as the original trace. 

However, we observe that such restrictions are unnecessary. For example, we can allow line {\tt 4} to receive its
value from line {\tt 1} as it has the same effect on thread local execution (i.e. the same branch is taken). 
However, the relaxation would allow us to execute lines {\tt 4} and {\tt 5} before line {\tt 2}, exposing 
the first missing race. We can also allow the trace to take the else branch such that the second race is also exposed. 
Note that even though the else branch was not executed originally, its effect can be precisely modeled as the value 
of $x$ is known from the original trace. As such, we can still perform sound analysis.

%that is missed, as it involves the statements in red, is between lines {\tt 2} and {\tt 7}.
Therefore in this paper, we propose \sysname, a relaxed predictive analysis. We identify the critical runtime
informantion that needs to be preserved during trace perturbation and relax the remaining information. 
In particular, we preserve all the addresses de-referenced (e.g. heap and array accesses) in the original execution,
and some of the branch outcomes such that part of the thread local paths stay intact. The criterion is that
we forbid \sysname to explore an unexecuted branch if the effect of the branch cannot be precisely modeled
(e.g. an array element is read while it was not defined in the original trace). The essence of our technique 
is to {\em explore the neighborhood of the original execution as much as possible}. Due to the substantially
enlarged search space, our results show that a race detector based on our relaxed predictive analysis 
can detect XXX times more races than existing predictive analysis, while guaranteeing soundness. 

Our contrituions are summarized as follows.
\begin{itemize}
\item We propose to relax existing preditive analysis and identify a bound of relaxation (i.e. what can
be relaxed).
\item We develop a constraint encoding scheme that uniformally encodes the strict information (i.e. the
part must be preserved) and the relaxed information. 
\item We develop a propotype \sysname. Our results show XXX.
\end{itemize}









\section{Basics}~\label{sec:basic}


\subsection{Trace Terminology}
Our analysis starts with a trace $\tau$,  a sequence of events, $e_0, e_1, \dots, e_n$.  
There are three types of events in general.
\begin{itemize}
\item the shared access, which includes the read and write of the shared fields, e.g., $o.f$=$x$ and $x$=$o.f$.
\item the local access, which includes only the access of the local variable, e.g., $x=y+z$ or $v.f=x$ (where $v$ is a thread-local object).
\item the branch event, which evaluates the branch condition to true/false, e.g., $x>3$.
\item the synchronization event, which includes start/join, wait/notify, lock/unlock events, e.g., $lock(o)$
\end{itemize}



Each event $e_i \in \Sigma$ is a tuple, $<t, id, a, v, ins>$, where $t\in \mathcal{T}$ denotes the thread generating the event, $id\in \mathcal{ID}$ denotes the unique integer assigned to the event (event id), $a\mathcal{A}$ denotes the address of the object or field (if any) accessed in the event,  $v\in \mathcal{V}$ denotes the value of the definition (if any) in the event, and $ins\in \mathcal{INS}$ denotes the three-address instruction generating the event.  Specifically, the address of the object $o$ is denoted as $id(o)\in \mathcal{ID}$, which is a string value representing $o$ uniquely, the address of the static field $f$ is  $id(f)$ and the address of the instance object field $o.f$ is  $id(o)\_id(f)$. Besides, as the event is derived by instrumenting the three-address code and monitoring the instrumented execution. Therefore, each event can involve at most three operands. 


The trace supports its standard operations as follows.
\begin{itemize}
\item projection, e.g., $\tau|t$ returns~\footnote{This is the abbreviation for the complete form $\tau|Thread=t$ } the events from the thread $t$,  $\tau|a$ returns the event involving the address $a$.  
\item concatenation. $\tau'=\tau e$ represents the new trace by appending the event $e$ to $\tau$.
\item length $|\tau|$. 
\item selecting an element. $\tau[0]$ and $\tau[|\tau|-1]$ represents the first and last event in $\tau$.
\end{itemize}


%The modeling of synchronization event is standard and explained in existing work, so we focus on the rest two types of events in this paper.

In addition, we maintain auxiliary information as follows.
\begin{itemize}
\item $AT: \mathcal{A} \times \mathcal{T} \rightarrow \gamma$ is a function that returns a trace $\tau \in \gamma$ that contains only the accesses of the address $a \in \mathcal{A}$ by the thread $t\in \mathcal{T}$. Each trace $\tau$ in $\gamma$ is defined over the alphabet of events $\Sigma$, specifically, the trace is an empty trace $\epsilon$ or defined in this way:  $\forall 0\leq i\leq |\tau|,   \tau[i]\in \Sigma, and, \forall i\neq j, \tau[i]\neq \tau[j]$. 
\item $R: \mathcal{A} \rightarrow \gamma$ is a function that returns the read accesses of the address $a\in \mathcal{A}$.
\item $W: \mathcal{A} \rightarrow \gamma$ is a function that returns the write accesses of the address $a \in \mathcal{A}$.
\item $Sync: \mathcal{A} \rightarrow \gamma$ is a function that returns the synchronization events involving the address $a\in \mathcal{A}$. 
\end{itemize}


\subsection{Symbolic Trace}
To facilitate the symbolic analysis, we need to introduce symbols to represent the operands in each event. Symbols allow us to overcome the limitation of concrete dependences and allow us to explore more dependences symbolically. 

%shared access only, local access only




{\bf Local Variables\  }
Like other symbolic analysis~\cite{jeff,chao}, the symbolic trace should be in the SSA form, i.e., each variable is defined exactly once. This is because the constraint solver employed by the analysis requires each variable to hold only one value. Besides, we need to make sure each use still reads from the same definition thread-locally. 

The  simple procedure shows the construction of the symbolic trace for local variables defined or used. The symbols are constructed by combining the static instruction and the runtime event id. Lines 6-10 handles the local variable definition.  We build a symbolic variable $s$ for it by combining the variable name and the event id. The uniqueness of the event id guarantees that each symbolic variable is defined exactly once. In addition, we replace the variable to the symbolic variable in the instruction and record the replacement in $table$. Lines 3-5 updates the local variable used in each event so that it is replaced with the symbolic variable for the corresponding definition. Here, the corresponding definition and the use share the same variable name, therefore, we can easily find out the symbolic variable through looking up the $table$. 

The SSA form of the trace is different from the SSA form of the instruction as the SSA instruction can only distinguish definitions at different program points but cannot distinguish the definitions at different execution points that share the same program point.





\begin{algorithmic}[3]
\For {$e: \tau$}
 \State $ins\gets e.ins$ 
 \For {$ins.use:ins.uses$}
 \State $ins.use \gets table(ins.use)$
 \EndFor
  \If {$ins.def\neq null$}
    \State $ s\gets ins.def^{e.id}$
	\State $ table [ins.def \rightarrow s]$
	\State $ins.def\gets s$
 \EndIf
\EndFor
\end{algorithmic}


%TODO update the intro+moti, make sure the same style
\begin{figure}
\centering
\begin{tabular}{ll}
\multicolumn{2}{c}{{\tt {\bf x} = 0; {\bf y} = 0;}} \\
\hline
\multicolumn{1}{c}{$T_1$} & \multicolumn{1}{c}{$T_2$} \\
\hline
{\tt 1: s=0; } & \\
{\tt 2: for(i=1;i<3;i++)} & \\
{\tt 3: \ \ \ s+=i;} & \\
{\tt 4: {\bf y} = s;} & \\
{\tt 5: {\bf x} = 1;} & \\
{\tt 6: {\bf y} = 5;} & \\
& {\tt 7: if ({\bf y} > 2)} \\
& {\tt 8:~~print({\bf x}+1);} \\	
& {\color{Gray} {\tt 9: else}} \\
& {\color{Gray} {\tt 10:~~print({\bf x}+2);}}
\end{tabular}
\caption{Running Example (shared variables are in bold font). }
\label{fig:running2}
\end{figure}



{\bf Shared Accesses}   Besides, we introduce symbols to represent shared reads and shared writes which leave the inter-thread dependence between reads/writes undetermined. For each read (or write) of shared variable $x$, we introduce $R^{id}_x$ (or $W^{id}_x$) to denote it, where $id$ is the event id.
Consider the code in Figure~\ref{fig:running2}, which resembles the example in Figure~\ref{fig:running} except that it includes a for loop at lines 1-2.
The symbolic trace is produced in Figure~\ref{fig:t4running2}. For simplicity, we only show the symbolic variables, while omitting other information such as thread information.




%TODO distinguish T1 and T2 in the text. 
\begin{figure}
\centering
\begin{tabular}{l|l}
\hline
\multicolumn{1}{c}{$Trace$} & \multicolumn{1}{c}{$Symbolic\  Trace$} \\
\hline
{\tt 0: {\bf x}=0} &  {\tt 0: $W^0_x$=0}    \\
{\tt 1: {\bf y}=0} &   {\tt 1: $W^1_y$=0}   \\
{\tt 2: s=0} &  {\tt 2: $s^2$=0}   \\
{\tt 3: i=1} &     {\tt 3: $i^3$=1}   \\
{\tt 4: i<3} &    {\tt 4: $i^3$<3} \\
{\tt 5: s=s+i} & {\tt 5: $s^5$=$s^2$+$i^3$}   \\
{\tt 6: i=2} &       {\tt 6: $i^6$=2}  \\
{\tt 7: i<3} &      {\tt 7: $i^6$<3}  \\
{\tt 8: s=s+i} &  {\tt 8: $s^8$=$s^5$+$i^6$}  \\
{\tt 9: i=3} &     {\tt 9: $i^9$=3}  \\
{\tt 10: i<3} &    {\tt 10: $i^9$<3}  \\
{\tt 11: {\bf y} = s;} &  {\tt 11: $W^{11}_y$ = $s^8$;}  \\
{\tt 12: {\bf x} = 1;} &    {\tt 12: $W^{12}_x$ = 1;}   \\
{\tt 13: {\bf y} = 5;} &    {\tt 13: $W^{13}_y$ = 5;}  \\
{\tt 14: {\bf y} > 2}  &    {\tt 14: $R^{14}_y$ > 2} \\
{\tt 15: tmp={\bf x}+1;}  & {\tt 15: tmp=$R^{15}_x$+1;}   \\	
{\tt 16: print(tmp);} &  {\tt 16: print(tmp);}  \\
\end{tabular}
\caption{Trace}
\label{fig:t4running2}
\end{figure}



{\bf Method Calls\ } The key to supporting the method calls is to capture the value flow the actual argument to the formal argument, and the flow from the return statement to the LHS variable of the method call. To explicitly model the value flow, record two additional events for each method call.
Consider the example in Figure~\ref{fig:methcall},   we record the local access event $y1=y;$ for the argument value flow and record the  local access event $x=i2;$. Recording the additional events is achieved through instrumenting the call site and the callee method statically.  

The above simple strategy however hides the complexity of the virtual method calls. At a call site of a virtual method, the static instrumentation cannot know precisely which method would be called. Therefore, we do not know what formal argument the actual argument flows to. Consider the example in Figure~\ref{fig:methcall}, suppose another implementation of the virtual method exists (in the comments).  We do not know how to instrument the code statically, $y1=y$ or $y2=y$. 


To avoid the problem, we have to combine the runtime knowledge. 
Our strategy is as follows: rather than directly record direct value flow from actual argument to formal argument, we introduce an artificial variable during the static instrumentation. Then we  insert the instrumentation $record(ARG0=y;)$ at call site, and insert  $record(y1=ARG0)$ at the entry of the method $func$ declared in the first class, and insert $record(y2=ARG0);$ at the entry of the method $func$ declared in the second class. At runtime, depending on which $func$ method is invoked, we record either the event sequence $ARG0=y; y1=ARG0$ or the sequence $ARG0=y; y2=ARG0$, which precisely captures the value flow. We model the return value flow similarly. 

Note that although different methods use the same names for the artificial variable,  they are translated to different variables after we get the SSA form of symbolic trace. 

 

\begin{figure}
\centering
\begin{tabular}{ll}
{\tt  x=o.func(y) } &  \\ 
 {\tt  func(y1)\{ // class O1} &  \\
 {\tt  \ \      i=y1; } & \\
 {\tt  \ \      i2=2*i; } &  \\
 {\tt  \  \      return i2;} & \\ 
 {\tt           \}} & \\ 
 
  {\tt //  func(y2)\{ // class O2} &  \\
 {\tt  // \ \      j=y2; } & \\
 {\tt  // \ \      j2=3*j; } &  \\
 {\tt  // \  \      return j;} & \\ 
 {\tt  //         \}} & \\ 
\end{tabular}
\caption{Method Calls}
\label{fig:methcall}
\end{figure}





\subsection{Constraints}
The constraints 












\section{Time Window}~\label{sec:basic}

\section{Heap Invariant}~\label{sec:basic}



% An example of a floating figure using the graphicx package.
% Note that \label must occur AFTER (or within) \caption.
% For figures, \caption should occur after the \includegraphics.
% Note that IEEEtran v1.7 and later has special internal code that
% is designed to preserve the operation of \label within \caption
% even when the captionsoff option is in effect. However, because
% of issues like this, it may be the safest practice to put all your
% \label just after \caption rather than within \caption{}.
%
% Reminder: the "draftcls" or "draftclsnofoot", not "draft", class
% option should be used if it is desired that the figures are to be
% displayed while in draft mode.
%
%\begin{figure}[!t]
%\centering
%\includegraphics[width=2.5in]{myfigure}
% where an .eps filename suffix will be assumed under latex,
% and a .pdf suffix will be assumed for pdflatex; or what has been declared
% via \DeclareGraphicsExtensions.
%\caption{Simulation Results}
%\label{fig_sim}
%\end{figure}

% Note that IEEE typically puts floats only at the top, even when this
% results in a large percentage of a column being occupied by floats.


% An example of a double column floating figure using two subfigures.
% (The subfig.sty package must be loaded for this to work.)
% The subfigure \label commands are set within each subfloat command, the
% \label for the overall figure must come after \caption.
% \hfil must be used as a separator to get equal spacing.
% The subfigure.sty package works much the same way, except \subfigure is
% used instead of \subfloat.
%
%\begin{figure*}[!t]
%\centerline{\subfloat[Case I]\includegraphics[width=2.5in]{subfigcase1}%
%\label{fig_first_case}}
%\hfil
%\subfloat[Case II]{\includegraphics[width=2.5in]{subfigcase2}%
%\label{fig_second_case}}}
%\caption{Simulation results}
%\label{fig_sim}
%\end{figure*}
%
% Note that often IEEE papers with subfigures do not employ subfigure
% captions (using the optional argument to \subfloat), but instead will
% reference/describe all of them (a), (b), etc., within the main caption.


% An example of a floating table. Note that, for IEEE style tables, the
% \caption command should come BEFORE the table. Table text will default to
% \footnotesize as IEEE normally uses this smaller font for tables.
% The \label must come after \caption as always.
%
%\begin{table}[!t]
%% increase table row spacing, adjust to taste
%\renewcommand{\arraystretch}{1.3}
% if using array.sty, it might be a good idea to tweak the value of
% \extrarowheight as needed to properly center the text within the cells
%\caption{An Example of a Table}
%\label{table_example}
%\centering
%% Some packages, such as MDW tools, offer better commands for making tables
%% than the plain LaTeX2e tabular which is used here.
%\begin{tabular}{|c||c|}
%\hline
%One & Two\\
%\hline
%Three & Four\\
%\hline
%\end{tabular}
%\end{table}


% Note that IEEE does not put floats in the very first column - or typically
% anywhere on the first page for that matter. Also, in-text middle ("here")
% positioning is not used. Most IEEE journals/conferences use top floats
% exclusively. Note that, LaTeX2e, unlike IEEE journals/conferences, places
% footnotes above bottom floats. This can be corrected via the \fnbelowfloat
% command of the stfloats package.



\section{Conclusion}
The conclusion goes here.




% conference papers do not normally have an appendix


% use section* for acknowledgement
\section*{Acknowledgment}


The authors would like to thank...





% trigger a \newpage just before the given reference
% number - used to balance the columns on the last page
% adjust value as needed - may need to be readjusted if
% the document is modified later
%\IEEEtriggeratref{8}
% The "triggered" command can be changed if desired:
%\IEEEtriggercmd{\enlargethispage{-5in}}

% references section

% can use a bibliography generated by BibTeX as a .bbl file
% BibTeX documentation can be easily obtained at:
% http://www.ctan.org/tex-archive/biblio/bibtex/contrib/doc/
% The IEEEtran BibTeX style support page is at:
% http://www.michaelshell.org/tex/ieeetran/bibtex/
%\bibliographystyle{IEEEtran}
% argument is your BibTeX string definitions and bibliography database(s)
%\bibliography{IEEEabrv,../bib/paper}
%
% <OR> manually copy in the resultant .bbl file
% set second argument of \begin to the number of references
% (used to reserve space for the reference number labels box)
%\begin{thebibliography}{1}
%\bibitem{IEEEhowto:kopka}
%H.~Kopka and P.~W. Daly, \emph{A Guide to \LaTeX}, 3rd~ed.\hskip 1em plus
%  0.5em minus 0.4em\relax Harlow, England: Addison-Wesley, 1999.
%\end{thebibliography}

\bibliographystyle{abbrv}
\bibliography{paper}  % sigproc.bib is the name of the Bibliography in this case


% that's all folks
\end{document}


