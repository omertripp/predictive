\documentclass[conference]{IEEEtran}

\usepackage{graphicx}
\usepackage{color}

\newcommand{\tool}{{\sc Recipe}}

\ifCLASSINFOpdf
  % \usepackage[pdftex]{graphicx}
  % declare the path(s) where your graphic files are
  % \graphicspath{{../pdf/}{../jpeg/}}
  % and their extensions so you won't have to specify these with
  % every instance of \includegraphics
  % \DeclareGraphicsExtensions{.pdf,.jpeg,.png}
\else
  % or other class option (dvipsone, dvipdf, if not using dvips). graphicx
  % will default to the driver specified in the system graphics.cfg if no
  % driver is specified.
  % \usepackage[dvips]{graphicx}
  % declare the path(s) where your graphic files are
  % \graphicspath{{../eps/}}
  % and their extensions so you won't have to specify these with
  % every instance of \includegraphics
  % \DeclareGraphicsExtensions{.eps}
\fi


\hyphenation{op-tical net-works semi-conduc-tor}

\usepackage{listings}

\lstset{
	numbers=left,
	numberstyle=\tiny,
	language={Java},
	mathescape=true,
	flexiblecolumns=true,
	morekeywords={def,Int,call,method,var,assert,share,unshare,acquire,release,fork,join,free,invariant,requires,ensures,acc,rd,old},
	basicstyle=\sffamily\small,
	moredelim=[is][\itshape]{@}{@},
	stepnumber=1,
	numbersep=2pt} 

\begin{document}
%
% paper title
% can use linebreaks \\ within to get better formatting as desired
\title{\tool: Relaxed Sound Predictive Analysis}


% author names and affiliations
% use a multiple column layout for up to three different
% affiliations
%\author{\IEEEauthorblockN{Michael Shell}
%\IEEEauthorblockA{School of Electrical and\\Computer Engineering\\
%Georgia Institute of Technology\\
%Atlanta, Georgia 30332--0250\\
%Email: http://www.michaelshell.org/contact.html}
%\and
%\IEEEauthorblockN{Homer Simpson}
%\IEEEauthorblockA{Twentieth Century Fox\\
%Springfield, USA\\
%Email: homer@thesimpsons.com}
%\and
%\IEEEauthorblockN{James Kirk\\ and Montgomery Scott}
%\IEEEauthorblockA{Starfleet Academy\\
%San Francisco, California 96678-2391\\
%Telephone: (800) 555--1212\\
%Fax: (888) 555--1212}}

% conference papers do not typically use \thanks and this command
% is locked out in conference mode. If really needed, such as for
% the acknowledgment of grants, issue a \IEEEoverridecommandlockouts
% after \documentclass

% for over three affiliations, or if they all won't fit within the width
% of the page, use this alternative format:
%
%\author{\IEEEauthorblockN{Michael Shell\IEEEauthorrefmark{1},
%Homer Simpson\IEEEauthorrefmark{2},
%James Kirk\IEEEauthorrefmark{3},
%Montgomery Scott\IEEEauthorrefmark{3} and
%Eldon Tyrell\IEEEauthorrefmark{4}}
%\IEEEauthorblockA{\IEEEauthorrefmark{1}School of Electrical and Computer Engineering\\
%Georgia Institute of Technology,
%Atlanta, Georgia 30332--0250\\ Email: see http://www.michaelshell.org/contact.html}
%\IEEEauthorblockA{\IEEEauthorrefmark{2}Twentieth Century Fox, Springfield, USA\\
%Email: homer@thesimpsons.com}
%\IEEEauthorblockA{\IEEEauthorrefmark{3}Starfleet Academy, San Francisco, California 96678-2391\\
%Telephone: (800) 555--1212, Fax: (888) 555--1212}
%\IEEEauthorblockA{\IEEEauthorrefmark{4}Tyrell Inc., 123 Replicant Street, Los Angeles, California 90210--4321}}




% use for special paper notices
%\IEEEspecialpapernotice{(Invited Paper)}




% make the title area
\maketitle


\begin{abstract}
%\boldmath
The abstract goes here.
\end{abstract}
% IEEEtran.cls defaults to using nonbold math in the Abstract.
% This preserves the distinction between vectors and scalars. However,
% if the conference you are submitting to favors bold math in the abstract,
% then you can use LaTeX's standard command \boldmath at the very start
% of the abstract to achieve this. Many IEEE journals/conferences frown on
% math in the abstract anyway.

% no keywords




% For peer review papers, you can put extra information on the cover
% page as needed:
% \ifCLASSOPTIONpeerreview
% \begin{center} \bfseries EDICS Category: 3-BBND \end{center}
% \fi
%
% For peerreview papers, this IEEEtran command inserts a page break and
% creates the second title. It will be ignored for other modes.
\IEEEpeerreviewmaketitle



%Limitations of static analysis: scalability when it is path sensitive, difficulty in modeling heap, cannot reason about concrete values, have to use
%abstraction, so that may not be sound, dynamic analysis is precise, easily scale to long and complex runs, but it only reasons about problems that
%did occur. 

%Predictive analysis is so good

%But it restricts too much, only relax schedule space and a little bit of the dependence space.

%We observe many dynamic aspects can be relaxed without affecting soundness.  as a result, instead of reasoning one
%execution, we can reason the neighborhood of the execution, in order words, we leverage dynamic analysis to provide
%the information that is difficult to analyze statically such as aliasing and a long execution path that covers the
%intended workflow, then we relax the predictive analysis to analyze neighborhood.  


\section{Introduction}\label{Se:introduction}
Program analyses can be largely classified to static and dynamic analyses. Static analyses abstract programs statically.
They are usually complete and do not require concrete program inputs to drive the analyses. However, they have difficulty in
scaling to large and complex programs when the analyses are path-sensitive, due to the sheer number of paths and the 
complexity of the paths. Path-insensitive analyses often do not have sufficient precision on the other hand. 
Static analyses also have difficulty in modeling complex aliasing behavior. In contrast, dynamic analysese focus 
on analyzing a few executions, usually just one. They are sound and easily scale to long and complex executions. But they 
can only reason about properties that manifest themselves in the execution(s). 

Recently, a novel kind of analysis called predictive analysis~\cite{...} was proposed. It has the capabilities of 
achieving soundness as dynamic analyses and reasoning about properties that did not occur during execution 
just like in static analysis. The basic idea is to leverage dynamic analysis to provide the information that 
is difficult to acquire through static analysis, such as a complex execution path that covers important functionalities 
and aliasing information. Note that during execution, we know precisely which write access reaches any given read 
access. While such information is encoded in a trace, predictive analysis reaons about mutation of the trace to 
expose problems or study properties that may not manifest during execution. It was used in sound data race detection. 
In particular, it collects a trace of threaded execution that contains rich runtime information such as concrete values, 
execution path, and memory accsses. It then leverages constraint solving to reason about if races can occur with
a different thread schedule, {\em while enforcing the critical runtime information (e.g. values, thread
local paths, and memory accesses) remains the same during schedule pertubation}. The rationale is that 
it becomes as difficult as static analysis if we allow these critical runtime information change. 
Preditive analysis has been shown to be very effective. It can detect real data races from complex programs.
And more importantly, it guarantees the results are sound (i.e. no false positives).

Despite its effectiveness, we observe that existing predictive analyes are too restrictive. They 
require too much runtime information to remain unchanged during analysis such that coverage is unnecessarily
limited. 


\begin{figure}
\centering
\begin{tabular}{ll}
\multicolumn{2}{c}{{\tt x = 0; y = 0;}} \\  % z = 3;
%\multicolumn{2}{c}{{\color{red} {\tt z = 2;}}} \\
\hline
\multicolumn{1}{c}{$T_1$} & \multicolumn{1}{c}{$T_2$} \\
\hline
{\tt 1: y = 3;} & \\
{\tt 2: x = 1;} & \\
{\tt 3: y = 5;} & \\
& {\tt 4: if (y > 2)} \\
& {\tt 5:~~z=1/x;} \\	
& {\color{red} {\tt 6: else}} \\
& {\color{red} {\tt 7:~~w=2/x;}}
\end{tabular}
\caption{Example illustrating ordering constraints beyond synchronization primitives}
\label{fig:running}
\end{figure}

Consider the example in Figure \ref{fig:running}, which shows an execution trace of two threads. The statements in red 
denote an unexplored branch outside the trace. Existing predictive analyses are unable to detect the race between 
lines {\tt 2} and {\tt 5} because they require that during trace permutations (by different schedules), a variable must not have
a different value in a permuted trace\footnote{They allow the variable involved in a race to have different values.}.
As such, they cannot execute line {\tt 4} before line {\tt 3}, which is needed to expose the race between lines {\tt 2}
and {\tt 5}. 
%the dependence between lines {\tt 3} and {\tt 4}, which is a barrier to the needed reorderings. 
A second race between lines {\tt 2} and {\tt 7} is also missed as the mutated trace must have the same thread local
path as the original trace. 

However, we observe that such restrictions are unnecessary. For example, we can allow line {\tt 4} to receive its
value from line {\tt 1} as it has the same effect on thread local execution (i.e. the same branch is taken). 
However, the relaxation would allow us to execute lines {\tt 4} and {\tt 5} before line {\tt 2}, exposing 
the first missing race. We can also allow the trace to take the else branch such that the second race is also exposed. 
Note that even though the else branch was not executed originally, its effect can be precisely modeled as the value 
of $x$ is known from the original trace. As such, we can still perform sound analysis.

%that is missed, as it involves the statements in red, is between lines {\tt 2} and {\tt 7}.
Therefore in this paper, we propose \sysname, a relaxed predictive analysis. We identify the critical runtime
informantion that needs to be preserved during trace perturbation and relax the remaining information. 
In particular, we preserve all the addresses de-referenced (e.g. heap and array accesses) in the original execution,
and some of the branch outcomes such that part of the thread local paths stay intact. The criterion is that
we forbid \sysname to explore an unexecuted branch if the effect of the branch cannot be precisely modeled
(e.g. an array element is read while it was not defined in the original trace). The essence of our technique 
is to {\em explore the neighborhood of the original execution as much as possible}. Due to the substantially
enlarged search space, our results show that a race detector based on our relaxed predictive analysis 
can detect XXX times more races than existing predictive analysis, while guaranteeing soundness. 

Our contrituions are summarized as follows.
\begin{itemize}
\item We propose to relax existing preditive analysis and identify a bound of relaxation (i.e. what can
be relaxed).
\item We develop a constraint encoding scheme that uniformally encodes the strict information (i.e. the
part must be preserved) and the relaxed information. 
\item We develop a propotype \sysname. Our results show XXX.
\end{itemize}









\section{Technical Overview}\label{sec:overview}

In this section, we walk the reader through a detailed technical description of our approach based on the example in Figure \ref{fig:running}. 
%
As input, we assume (i) a program $P$ as well as (ii) a trace of $P$ recorded during a dynamic execution.

% in Static Single Assignment (SSA) form, such that every variable is defined exactly once.


\subsection{Preliminaries}
To facilitate our presentation, we first introduce some terminologies used throughout this paper. Intuitively, a trace is a sequence of events recorded during the observation run.




{\bf Event\ } An event, $e=<t, id, inst>$, is a concrete representation of the runtime execution of a static instruction $inst$.


\begin{itemize}
\item $t$ refers to the thread that issues the event $e$, denoted as $t^e$.
\item  $id$ refers to the id associated with each event. The key property of  $id$ is {\em uniqueness}, i.e., any two events in the trace own different ids.  Unless otherwise specified, we use  the index of an event in the trace as its id, which satisfies the above property. Throughout this paper,  we denote an event as $e_{id}$ with the id as the subscript. %We may use the terms $e_{id}$ and $id$ interchangeably given their one-to-one correspondence, e.g., $t^{e_3}$ and $t^3$. 
\item $inst$ is the static instruction. The instructions are three-address instructions involving at most three operands, which modern compilers commonly support.  Specifically, we are interested in the types of instructions listed in Table~\ref{Ta:syntax}. When the variable does not appear on the left hand of an equation, such as $y$ in $x.f=y$, it may refer to a variable, a constant or event object creation expression $new (...)$.  The $bop$ stands for the binary operator, which may refer to $+, -, *, /, \%, \wedge, \vee$ in the assignment, or refer to $<, >, =, \wedge, \vee$ in the branch. The target of the branch event is not important in our scope, therefore, we may abbreviate the branch instruction as the boolean expression afterwards. The listed instructions suffice to represent all trace events of interest. This is because a concrete finite execution trace can be reduced to a straight-line loop-free call-free path program  (argument passing of the method call is modeled as assignments and the virtual call resolution is modeled as branches). This standard form of simplification preserves all the data-race-related information contained in the original trace. 
\end{itemize}



\begin{table}
	\begin{center}
		\begin{tabular}{rcl}
			\multicolumn{1}{l}{{\tt s} $::=$} & & \\
			{\tt y = x.f} & $|$ & {\bf (heapr)} \\ 
			{\tt x.f = y}  & $|$ & {\bf (heapw)} \\ %\ $|$\ {\tt x.f = $c$}\
			{\tt z = x $bop$ y}\  & $|$ & {\bf (assign)} \\ %$|$\ {\tt z = $c$} $|$ {\tt z = new()}
			{\tt if (x $bop$  y) goto ...} & $|$ &  {\bf (branch)} \\
			{\tt lock(l)}\ $|$\ {\tt unlock(l)}  & $|$& {\bf (sync)} \\
			{\tt fork(t)}\ $|$\ {\tt join(t)}$|$ {\tt begin(t)}\ $|$\ {\tt end(t)} &  & {\bf (thread)}
		\end{tabular}
	\end{center}
	\caption{\label{Ta:syntax}Instruction Types}
\end{table}


{\bf Trace \ } A trace $\tau$ is a sequence of events. It can also be represented more comprehensively as,  $\tau=<\Gamma , \{\tau_{t_1}, \tau_{t_2}, \dots \tau_{t_n} \}, O, \theta>$.
\begin{itemize}
\item  $\Gamma$ refers to the set of threads involved in the trace, which include $t_1$, $t_2$, \dots, $t_n$.
\item   $\{\tau_{t_1}, \tau_{t_2}, \dots \tau_{t_n} \}$ denotes the event sequences produced by each thread.  
\item  $O$ is a function that assigns  to every event an integer value that denotes its scheduling order. For example, $e_i$ is scheduled earlier than $e_j$ if $O(e_i)<O(e_j)$.
\item  $\theta$ is a function that maps every variable to a value. For example, given an event $e$ executing the instruction $x=y+z$, the mapping recorded in the trace may be $\theta(x)=3$,  $\theta(y)=2$,  $\theta(z)=1$.  If the variable $x$ is defined in two events, the mapping for $x$ becomes ambiguous. To avoid the ambiguity, we conduct a pre-processing  (Section~\ref{sec:relax1}) that produces the SSA form of the trace, in which each variable is defined exactly once. 
\end{itemize}

%\item $map$ is a mapping from live expressions at the current event to their value (and thus a partial mapping from expressions to values). Live expressions include program variables, object fields, boolean expressions, etc. We use the notation $\lsyn exp \rsyn^e$ to retrieve the value associated with expression $exp$ at $e$.
%Specially,  we also store the heap location value  $\lsyn \overline{x.f} \rsyn^e$ for a field $x.f$. Note the difference between $\lsyn \overline{x.f} \rsyn^e$ and $\lsyn x.f \rsyn^e$, which represent the location and the value stored in it respectively. The location value $\lsyn \overline{x.f} \rsyn^e$ is computed as a pair $(\lsyn x \rsyn^e, f)$.


%We introduce the projection operation ``$\downarrow$'' to facilitate the reasoning of the trace:
%$\tau\downarrow_{t}$ contains only the (ordered) events from the thread $t$; $\tau\downarrow_{\ell}$ contains the events involving the location $\ell$; $\tau\downarrow_{\leq id}$ denotes the prefix of $e_{id}$, i.e., the events preceding the event $e_{id}$; $\tau\downarrow_{\geq id}$ denotes the events after $e_{id}$; $\tau\downarrow_{>=id1\wedge <=id2}$ denotes the events between $e_{id1}$ and $e_{id2}$.

%\begin{itemize}
%\item projection operation  : 
%%\item sets: $\mathcal{T}$ denotes the set of threads involved;  $\mathcal{L}$ denotes the set of memory locations involved; 
%%$\mathcal{SV}$ denotes the set of shared variables, which reference the shared locations, the shared location $l$ can be judged by counting the number of threads in $\tau\downarrow_{\ell}$;  $\mathcal{E}$ denotes the set of events involved.
%\end{itemize}






%We make use of the following helper functions:
%\begin{itemize}
%	\item ${\sf proj}\ t\ i$ projects trace $t$ onto all transitions involving thread $i$.
%	\item $t[k]$ obtains the $k$th transition within trace $t$.
%	\item ${\sf index}\ t\ \tau$ retrieves the index, or offset, of transition $\tau$ within trace $t$. When simply writing
%	${\sf index}\ \tau$ (while omitting the trace parameter) we refer to the index of $\tau$ within the original trace. 
%	\item ${\sf pre}\ t\ \tau$ is the prefix of trace $t$ preceding transition $\tau$. For the suffix beyond $\tau$, we 
%	use ${\sf post}\ t\ \tau$. Finally, ${\sf bet}\ t\ \tau_1\ \tau_2$ returns the transitions delimited by $\tau_1$ and $\tau_2$.
%\end{itemize}

%Throughout this paper, we assume a standard operational semantics, which defines (i) a mapping ${\sf thr}$ between
%execution threads, each having a unique identifier $i \in \mathbb{N}$, and their respective code, as well as (ii) per-thread stack and shared heap memory. 

%The code executed by a given thread follows the syntax in Table \ref{Ta:syntax}. The text of a program is a sequence of zero or more method declarations, as given in the definition of symbol {\tt p}. Methods accept zero or more arguments $\overline{{\tt x}}$, have a body ${\tt s}$, and may have a return value (which we leave implicit). For simplicity, we avoid from static typing as well as virtual methods. The body of a method consists of the core grammar for symbol {\tt s}. We avoid from specifying syntax checking rules, as the grammar is fully standard.

%For simplicity, we assume that in the starting state each thread points to a parameter-free method. In this way, we can simply assume an empty starting state (i.e., an empty heap), and eliminate complexities such as user-provided inputs and initialization of arguments with default values. These extensions are of course possible, and in practice we handle these cases, but reflecting them in the formalism would result in needless complications.

%As is standard, we assume an interleaved semantics of concurrency. A \emph{transition}, or \emph{event}, is of the form
%$\sigma \stackrel{i / {\tt s}}{\longrightarrow} \sigma'$, denoting that thread $i$ took an evaluation step
%in prestate $\sigma$, wherein atomic
%statement {\tt s} was executed, which resulted in poststate $\sigma'$. We refer to a sequence of transitions from
%the starting state to either an exceptional state (e.g., due to null dereference) or a state where all the threads have 
%reduced their respective code to $\epsilon$ as a \emph{trace}.


%TODO this section is inconsistent with others. change it!
\subsection{Constraint System}
The predictive analysis takes the program and a trace as the input.  Suppose it starts with the following trace of the program in  Figure \ref{fig:running}:  $e_1 e_2 e_3 e_4 e_5$, where the line number is used as the event id (subscript). A potential race is between the pair, $(e_2, e_5)$, as they access the shared variable $x$ from different threads and one is a write. 

Given the trace, $\tau=<\Gamma , \{\tau_{t_1}, \tau_{t_2}, \dots \tau_{t_n} \}, O, \theta>$, the predictive analysis computes a new trace,  $\tau'=<\Gamma , \{\tau_{t_1}, \tau_{t_2}, \dots \tau_{t_n} \}, O', \theta'>$. $\tau'$ preserves the same event sequence for each thread, i.e., $T_1$ still executes $e_1 e_2 e_3$ and $T_2$ still executes $e_4 e_5$, but it reschedules the events from different threads so that the potential race pair runs concurrently, i.e., $e_2$ and $e_5$ run concurrently. When the schedule changes, the variables may get different values. Therefore, we need to compute both the new schedule $O'$ and the new value mapping $\theta'$, in order to witness a race. To guarantee that $\tau'$ is a feasible trace of the program $P$, the computation needs to respect a set of constraints, which include the value constraints, control flow constraints and synchronization constraints~\footnote{We omit the synchronization constraints in this paper, which are already well explained in existing techniques~\cite{yannis, pecan}.}. We explain them briefly in the following. The full details are explained in Section~\ref{sec:relax1} and the feasibility guarantee is explained in Section~\ref{sec:guarantee}.




%The local accesses are not used in this example and therefore omitted.






%We begin with an explanation of the encoding process. The resulting formula is provided in Figure \ref{fig:encoding}. We describe the conjuncts comprising the formula one by one.

%\begin{figure*}
%	\begin{center}
%$$
%	\begin{array}{rcl}
%	& \left( O_1 < O_2 < O_3 \wedge O_4 < O_5 \wedge O_4 < O_7 \right) & \textbf{(program order)} \\
%\bigwedge & \left( W_{\tt x}^0 = 0 \wedge W_{\tt y}^0 = 0 \wedge W_{\tt z}^0 = 3 \wedge W_{\tt y}^1 = 3 \wedge 
%	W_{\tt x}^2 = 1 \wedge W_{\tt y}^3 = 5 \right) & \textbf{(variable definitions)} \\
%\bigwedge & \left(		(R_{\tt y}^4=W_{\tt y}^0 \wedge O_4 < O_1) \vee	
%(R_{\tt y}^4=W_{\tt y}^1 \wedge O_1 < O_4 < O_3) \vee
%(R_{\tt y}^4=W_{\tt y}^3 \wedge O_3 < O_4)
%		\right) & \textbf{(thread interference)} \\
%\bigwedge & R_{\tt y}^4 > 2 \equiv {\bf true} & \textbf{(path conditions)} \\
%\bigwedge & t^2 \neq t^5 \wedge (2\ writes\ {\tt x} \vee 5\ writes\ {\tt x}) \wedge O_2 = O_5 & \textbf{(race condition)}
%	\end{array} 
%$$
%\end{center}
%\caption{\label{fig:encoding}\tool\ encoding of the trace in Figure \ref{fig:running} as a constraint system}
%\end{figure*}


We first preprocess the trace as follows. We introduce symbols to replace the shared accesses in the instructions, e.g., We use $W^{id}_{x}$/$R^{id}_{x}$ to replace the write/read of the shared variable $x$ by the event $e_{id}$. %The rationale is explained in Section~\ref{sec:relax1}.

{\bf Race Condition\ } The potential race pair involves two accesses of the same shared location from different threads, where at least one access is a write. Given any potential race pair, e.g., ($e_2$, $e_5$), it is a real race if and only if the two accesses can occur at the same time (race condition), which is captured by the race condition constraint
$$
	O'(e_2) = O'(e_5)
$$.
The race condition constraint--- together with the other constraints --- guarantees the feasibility of the predicted race if a solution is found for the overall constraint system.

{\bf Intra-thread Constraints\ } The predictive analysis needs to preserve the same event sequence for each thread, which further requires 

\begin{itemize}
\item {\bf Control Flow Constraints\ } Each thread takes the same control flows (or branch decisions) to reproduce the events.
 For the running example (ignoring the statements in red), this yields:
$$
	\theta'(R_{\tt y}^4) > 2 \equiv {\bf true}
$$
That is, in the predicted trace $\tau'$, the value of variable {\tt y} read at $e_4$ should be greater than $2$. 


\item {\bf Intra-thread Order Constraints\ } The events should follow the same thread-local order as in the original trace. This constraint is imposed by the fact that the two runs share the same instruction sequence and take the same control flows. 
We yield the following formula:
$$
	O'(e_1) < O'(e_2) < O'(e_3) \wedge O'(e_4) < O'(e_5)
$$
That is, in the predicted trace $\tau'$, the three events from $T_1$ are ordered in the same way as in $\tau$, and similarly, the branch event $e_4$ in $T_2$ still executes before the event $e_5$ inside the branch body.

\item {\bf Intra-thread Value Constraints\ } The events should respect the value constraints imposed by each instruction.
 For our running example, we obtain:
$$
\begin{array}{l}
	\theta'(W_{\tt x}^0) = 0 \wedge \theta'(W_{\tt y}^0) = 0 \wedge \theta'(W_{\tt y}^1) = 3 \\ 
	\wedge	\theta'(W_{\tt x}^2) = 1 \wedge \theta'(W_{\tt y}^3) = 5
	\end{array}
$$
As an example, $\theta'(W_{\tt y}^3)= 5$ denotes that the predicted run still assigns the value 5 to the variable {\tt y}  at event $e_3$.
\end{itemize}

%The next set of constraints, denoted ${\tt w}^i=k$ for local variables and $W_{\tt w}^i=k$ for shared variables, capture variable definitions: Variable {\tt w} is assigned value $k$ at statement $i$.


%Indeed, this constraint is satisfied by the assignments to {\tt y} both at line {\tt 1} and at line {\tt 3}.


{\bf Inter-thread Value Constraints\ } The inter-thread value constraints capture what writes the read events read from and under what scheduling condition the value flow occurs.  The resulting formula for our example is
$$
\begin{array}{lcl}
	& & (\theta'(R_{\tt y}^4)=\theta'(W_{\tt y}^0) \wedge O'(e_4) < O'(e_1) ) \\
& \bigvee &
	(\theta'(R_{\tt y}^4)=\theta'(W_{\tt y}^1) \wedge O'(e_1) < O'(e_4) < O'(e_3)) \\
& \bigvee &
	(\theta'(R_{\tt y}^4)=\theta'(W_{\tt y}^3) \wedge O'(e_3) < O'(e_4))
\end{array}
$$    
Notice, importantly, that the formula associates the value constraints with the scheduling order constraints. 
 As an example, 
$\theta'(R_{\tt y}^4)=\theta'(W_{\tt y}^1) \wedge O'(e_1) < O'(e_4) < O'(e_3)$ means that, in the predicted run, the read $e_4$ reads from the write $e_1$, under the condition that $e_4$ happens after $e_1$ and no other  writes (such as $e_3$) interleave them.



{\bf Unexplored Branches\ } Beyond the encoding steps so far, which focus on the given trace, we can often encode constraints along unexplored branches. In our running example, this is essential to discover the race between lines {\tt 2} and {\tt 7}. We conduct the symbolic execution to exercise the unexplored branches, which returns a set of traces representing possible executions of the unexplored branch. Each trace is combined with the original trace to find races involving the accesses in the unexplored path. For the racy events, $e_2$ and $e_7$, we first specify the race condition constraint  $O'(e_2)=O'(e_7)$. In addition, we negate the path condition, thereby obtaining $\theta'(R_{\tt y}^4) > 2 \equiv {\bf false}$ in place of $\theta'(R_{\tt y}^4) > 2 \equiv {\bf true}$. We also model the other necessary constraints such as the intra-thread order constraint $O'(e_4) < O'(e_7)$, similar to the above analysis.

{\bf Constraint Solving\ } The formuals from the different encoding steps are conjoined and sent to the off-the-shelf solver such as {\sf z3}.  The race under analysis is real if the solver returns a solution, which includes the scheduling order and the values for the variables. 

{\bf Highlights \ } In this example, our analysis finds two more pairs of races than existing analyses~\cite{yannis, jeff}, which include $(e_2, e_5)$ and $(e_2, e_7)$.
The key insight for finding the first race is, we do not require $e_4$ to read from $e_3$ to retain the original value, as what existing approaches do, rather, we allow $e_4$ to read from $e_1$ while still preserving the feasibility of the branch. This relaxation allows the events after $e_4$ to run concurrently with the events before $e_3$, leading to the detection of the first race.  The key insight for finding the second race is, we explore the unexplored branches and ask the solver to compute a feasible schedule that exercises them. The neighboring branches are likely to contain racy events if the executed branches contain racy events.

 
%In particular, the solution discloses a feasible trace that gives rise to the race at hand. The trace is identified uniquely via the order enforced in the solution over the variables $O_i$, which represent scheduling order. 



% An example of a floating figure using the graphicx package.
% Note that \label must occur AFTER (or within) \caption.
% For figures, \caption should occur after the \includegraphics.
% Note that IEEEtran v1.7 and later has special internal code that
% is designed to preserve the operation of \label within \caption
% even when the captionsoff option is in effect. However, because
% of issues like this, it may be the safest practice to put all your
% \label just after \caption rather than within \caption{}.
%
% Reminder: the "draftcls" or "draftclsnofoot", not "draft", class
% option should be used if it is desired that the figures are to be
% displayed while in draft mode.
%
%\begin{figure}[!t]
%\centering
%\includegraphics[width=2.5in]{myfigure}
% where an .eps filename suffix will be assumed under latex,
% and a .pdf suffix will be assumed for pdflatex; or what has been declared
% via \DeclareGraphicsExtensions.
%\caption{Simulation Results}
%\label{fig_sim}
%\end{figure}

% Note that IEEE typically puts floats only at the top, even when this
% results in a large percentage of a column being occupied by floats.


% An example of a double column floating figure using two subfigures.
% (The subfig.sty package must be loaded for this to work.)
% The subfigure \label commands are set within each subfloat command, the
% \label for the overall figure must come after \caption.
% \hfil must be used as a separator to get equal spacing.
% The subfigure.sty package works much the same way, except \subfigure is
% used instead of \subfloat.
%
%\begin{figure*}[!t]
%\centerline{\subfloat[Case I]\includegraphics[width=2.5in]{subfigcase1}%
%\label{fig_first_case}}
%\hfil
%\subfloat[Case II]{\includegraphics[width=2.5in]{subfigcase2}%
%\label{fig_second_case}}}
%\caption{Simulation results}
%\label{fig_sim}
%\end{figure*}
%
% Note that often IEEE papers with subfigures do not employ subfigure
% captions (using the optional argument to \subfloat), but instead will
% reference/describe all of them (a), (b), etc., within the main caption.


% An example of a floating table. Note that, for IEEE style tables, the
% \caption command should come BEFORE the table. Table text will default to
% \footnotesize as IEEE normally uses this smaller font for tables.
% The \label must come after \caption as always.
%
%\begin{table}[!t]
%% increase table row spacing, adjust to taste
%\renewcommand{\arraystretch}{1.3}
% if using array.sty, it might be a good idea to tweak the value of
% \extrarowheight as needed to properly center the text within the cells
%\caption{An Example of a Table}
%\label{table_example}
%\centering
%% Some packages, such as MDW tools, offer better commands for making tables
%% than the plain LaTeX2e tabular which is used here.
%\begin{tabular}{|c||c|}
%\hline
%One & Two\\
%\hline
%Three & Four\\
%\hline
%\end{tabular}
%\end{table}


% Note that IEEE does not put floats in the very first column - or typically
% anywhere on the first page for that matter. Also, in-text middle ("here")
% positioning is not used. Most IEEE journals/conferences use top floats
% exclusively. Note that, LaTeX2e, unlike IEEE journals/conferences, places
% footnotes above bottom floats. This can be corrected via the \fnbelowfloat
% command of the stfloats package.



\section{Conclusion}
The conclusion goes here.




% conference papers do not normally have an appendix


% use section* for acknowledgement
\section*{Acknowledgment}


The authors would like to thank...





% trigger a \newpage just before the given reference
% number - used to balance the columns on the last page
% adjust value as needed - may need to be readjusted if
% the document is modified later
%\IEEEtriggeratref{8}
% The "triggered" command can be changed if desired:
%\IEEEtriggercmd{\enlargethispage{-5in}}

% references section

% can use a bibliography generated by BibTeX as a .bbl file
% BibTeX documentation can be easily obtained at:
% http://www.ctan.org/tex-archive/biblio/bibtex/contrib/doc/
% The IEEEtran BibTeX style support page is at:
% http://www.michaelshell.org/tex/ieeetran/bibtex/
%\bibliographystyle{IEEEtran}
% argument is your BibTeX string definitions and bibliography database(s)
%\bibliography{IEEEabrv,../bib/paper}
%
% <OR> manually copy in the resultant .bbl file
% set second argument of \begin to the number of references
% (used to reserve space for the reference number labels box)
%\begin{thebibliography}{1}
%\bibitem{IEEEhowto:kopka}
%H.~Kopka and P.~W. Daly, \emph{A Guide to \LaTeX}, 3rd~ed.\hskip 1em plus
%  0.5em minus 0.4em\relax Harlow, England: Addison-Wesley, 1999.
%\end{thebibliography}

\bibliographystyle{abbrv}
\bibliography{paper}  % sigproc.bib is the name of the Bibliography in this case


% that's all folks
\end{document}


