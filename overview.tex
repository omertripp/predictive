\section{Technical Overview}\label{sec:overview}

In this section, we walk the reader through a detailed technical description of our approach based on the example in Figure \ref{fig:running}. 
%
As input, we assume (i) a program $P$ as well as (ii) a trace of $P$ recorded during a dynamic execution.

% in Static Single Assignment (SSA) form, such that every variable is defined exactly once.


\subsection{Preliminaries}
We first introduce some notation used throughout this paper. Intuitively, a 
trace is a sequence of events recorded during the observed run. An 
event, $e=<t, id, inst>$, is a concrete representation of the runtime 
execution of a static instruction $inst$, such that (i) $t$ refers to the 
thread; (ii) $id$ is a {\em unique} identifier of the event (unless specified 
otherwise, we use the index of an event in the trace as its id); 
and (iii) $inst$ is a static three-address instruction (involving at most three operands).


For this paper, we utilize the instructions listed in 
Table~\ref{Ta:syntax}. When the variable does not appear on the left hand 
side of an equation, such as $y$ in $x.f=y$, it may refer to a variable, a 
constant or an object creation expression $new (...)$.  Symbol $bop$ stands 
for a binary operator, which may refer to $+, -, *, /, \%, \wedge, \vee$ in 
an assignment, or  $<, >, =, \wedge, \vee$ in a branch instruction. 
The target of the branch instruction is not important in our scope, 
therefore, we may abbreviate a branch instruction as its boolean expression 
afterwards. The listed instructions suffice to represent all trace events 
of interest. 
Note that 
%This is because 
a concrete finite execution trace can be reduced to a straight-line 
loop-free call-free path program  (argument passing of a method call 
is modeled as assignments and virtual call resolutions are modeled as 
branches). 
%This standard form of simplification preserves all the data-race-related information contained in the original trace. 

\begin{table}
{ 
	\begin{center}
		\begin{tabular}{rcl}
			\multicolumn{1}{l}{{\tt s} $::=$} & & \\
			{\tt y = x.f} & $|$ & {\bf (heapr)} \\ 
			{\tt x.f = y}  & $|$ & {\bf (heapw)} \\ %\ $|$\ {\tt x.f = $c$}\
			{\tt z = x $bop$ y}\  & $|$ & {\bf (assign)} \\ %$|$\ {\tt z = $c$} $|$ {\tt z = new()}
			{\tt if (x $bop$  y) goto ...} & $|$ &  {\bf (branch)} \\
			{\tt lock(l)}\ $|$\ {\tt unlock(l)}  & $|$& {\bf (sync)} \\
			{\tt fork(t)}\ $|$\ {\tt join(t)}$|$ {\tt begin(t)}\ $|$\ {\tt end(t)} &  & {\bf (thread)}
		\end{tabular}
	\end{center}
	}
	\caption{\label{Ta:syntax}Instruction Types}
\vspace{-3em}
\end{table}


{\bf Trace Representation \ } A trace $\tau$ is represented as $\tau=<\Gamma , \{\tau_{t_1}, \tau_{t_2}, \dots \tau_{t_n} \}, O, \theta>$, where
\begin{itemize}
\item  $\Gamma$ refers to the set of threads involved in the trace, which include $t_1$, $t_2$, \dots, $t_n$;
\item   $\{\tau_{t_1}, \tau_{t_2}, \dots \tau_{t_n} \}$ denotes the event sequences produced by each thread;
\item  $O$ is a function that assigns  to every event an integer value that denotes its scheduling order: $e_i$ is scheduled earlier than $e_j$ iff $O(e_i)<O(e_j)$; and
\item  $\theta$ is a function that maps every live variable to a value. 
%For example, given an event $e$ executing the instruction $x=y+z$, the mapping recorded in the trace may be $\theta(x)=3$,  $\theta(y)=2$,  $\theta(z)=1$.  
\end{itemize}
To ensure the integrity of $\theta$ as well simplify downstream encoding 
steps,
% performed by the analysis already at the syntactic level, 
we rewrite the raw input trace into SSA form, such that each variable is defined exactly once.


%\item $map$ is a mapping from live expressions at the current event to their value (and thus a partial mapping from expressions to values). Live expressions include program variables, object fields, boolean expressions, etc. We use the notation $\lsyn exp \rsyn^e$ to retrieve the value associated with expression $exp$ at $e$.
%Specially,  we also store the heap location value  $\lsyn \overline{x.f} \rsyn^e$ for a field $x.f$. Note the difference between $\lsyn \overline{x.f} \rsyn^e$ and $\lsyn x.f \rsyn^e$, which represent the location and the value stored in it respectively. The location value $\lsyn \overline{x.f} \rsyn^e$ is computed as a pair $(\lsyn x \rsyn^e, f)$.


%We introduce the projection operation ``$\downarrow$'' to facilitate the reasoning of the trace:
%$\tau\downarrow_{t}$ contains only the (ordered) events from the thread $t$; $\tau\downarrow_{\ell}$ contains the events involving the location $\ell$; $\tau\downarrow_{\leq id}$ denotes the prefix of $e_{id}$, i.e., the events preceding the event $e_{id}$; $\tau\downarrow_{\geq id}$ denotes the events after $e_{id}$; $\tau\downarrow_{>=id1\wedge <=id2}$ denotes the events between $e_{id1}$ and $e_{id2}$.

%\begin{itemize}
%\item projection operation  : 
%%\item sets: $\mathcal{T}$ denotes the set of threads involved;  $\mathcal{L}$ denotes the set of memory locations involved; 
%%$\mathcal{SV}$ denotes the set of shared variables, which reference the shared locations, the shared location $l$ can be judged by counting the number of threads in $\tau\downarrow_{\ell}$;  $\mathcal{E}$ denotes the set of events involved.
%\end{itemize}






%We make use of the following helper functions:
%\begin{itemize}
%	\item ${\sf proj}\ t\ i$ projects trace $t$ onto all transitions involving thread $i$.
%	\item $t[k]$ obtains the $k$th transition within trace $t$.
%	\item ${\sf index}\ t\ \tau$ retrieves the index, or offset, of transition $\tau$ within trace $t$. When simply writing
%	${\sf index}\ \tau$ (while omitting the trace parameter) we refer to the index of $\tau$ within the original trace. 
%	\item ${\sf pre}\ t\ \tau$ is the prefix of trace $t$ preceding transition $\tau$. For the suffix beyond $\tau$, we 
%	use ${\sf post}\ t\ \tau$. Finally, ${\sf bet}\ t\ \tau_1\ \tau_2$ returns the transitions delimited by $\tau_1$ and $\tau_2$.
%\end{itemize}

%Throughout this paper, we assume a standard operational semantics, which defines (i) a mapping ${\sf thr}$ between
%execution threads, each having a unique identifier $i \in \mathbb{N}$, and their respective code, as well as (ii) per-thread stack and shared heap memory. 

%The code executed by a given thread follows the syntax in Table \ref{Ta:syntax}. The text of a program is a sequence of zero or more method declarations, as given in the definition of symbol {\tt p}. Methods accept zero or more arguments $\overline{{\tt x}}$, have a body ${\tt s}$, and may have a return value (which we leave implicit). For simplicity, we avoid from static typing as well as virtual methods. The body of a method consists of the core grammar for symbol {\tt s}. We avoid from specifying syntax checking rules, as the grammar is fully standard.

%For simplicity, we assume that in the starting state each thread points to a parameter-free method. In this way, we can simply assume an empty starting state (i.e., an empty heap), and eliminate complexities such as user-provided inputs and initialization of arguments with default values. These extensions are of course possible, and in practice we handle these cases, but reflecting them in the formalism would result in needless complications.

%As is standard, we assume an interleaved semantics of concurrency. A \emph{transition}, or \emph{event}, is of the form
%$\sigma \stackrel{i / {\tt s}}{\longrightarrow} \sigma'$, denoting that thread $i$ took an evaluation step
%in prestate $\sigma$, wherein atomic
%statement {\tt s} was executed, which resulted in poststate $\sigma'$. We refer to a sequence of transitions from
%the starting state to either an exceptional state (e.g., due to null dereference) or a state where all the threads have 
%reduced their respective code to $\epsilon$ as a \emph{trace}.



\subsection{Constraint System}
In this section, we are going to informally explain the two relaxations
we make which allow us to catch the two races missed by existing 
predictive analyses.
Our analysis takes the program and a trace as the input.  
Suppose it starts with the following trace of the program in  
Figure \ref{fig:running}:  $e_1 e_2 e_3 e_4 e_5$, where the line number 
is used as the event id (subscript). 

\smallskip
\noindent
{\em I. Value Relaxation\ } Our first relaxation is to allow variables
to have different values as long as the same thread local path is preserved.
It does not require data dependences or same variable values as in~\cite{yannis, pldi14, Said:2011}.
This allows us to catch the racy pair $(e_2, e_5)$.
%A potential race is between 
%the pair, $(e_2, e_5)$, as they access the shared variable $x$ from 
%different threads and one is a write. 
Given a trace, $\tau=<\Gamma , \{\tau_{t_1}, \tau_{t_2}, \dots \tau_{t_n} \},
 O, \theta>$, our analysis computes a new trace,  
$\tau'=<\Gamma , \{\tau_{t_1}, \tau_{t_2}, \dots \tau_{t_n} \}, O', \theta'>$.
 $\tau'$ preserves the same event sequence for each thread, i.e., $T_1$ still 
executes $e_1 e_2 e_3$ and $T_2$ still executes $e_4 e_5$, but it reschedules 
the events from different threads so that the potential race pair runs 
concurrently, i.e., $e_2$ and $e_5$ run concurrently. When the schedule 
changes, the variables may get different values. Therefore, we need to 
compute both the new schedule $O'$ and the new value mapping $\theta'$, in 
order to witness a race. To guarantee that $\tau'$ is a feasible trace of 
the program $P$, the computation needs to respect a set of constraints, 
which include the value constraints, control flow constraints and 
synchronization constraints~\footnote{We omit the synchronization constraints 
in this paper, which are already well explained in existing 
techniques~\cite{yannis, pldi14}.}. We explain the constraints 
briefly in the following. Details are in Section~\ref{sec:relax1} 
and the soundness guarantee is explained in Section~\ref{sec:guarantee}.




%The local accesses are not used in this example and therefore omitted.






%We begin with an explanation of the encoding process. The resulting formula is provided in Figure \ref{fig:encoding}. We describe the conjuncts comprising the formula one by one.

%\begin{figure*}
%	\begin{center}
%$$
%	\begin{array}{rcl}
%	& \left( O_1 < O_2 < O_3 \wedge O_4 < O_5 \wedge O_4 < O_7 \right) & \textbf{(program order)} \\
%\bigwedge & \left( W_{\tt x}^0 = 0 \wedge W_{\tt y}^0 = 0 \wedge W_{\tt z}^0 = 3 \wedge W_{\tt y}^1 = 3 \wedge 
%	W_{\tt x}^2 = 1 \wedge W_{\tt y}^3 = 5 \right) & \textbf{(variable definitions)} \\
%\bigwedge & \left(		(R_{\tt y}^4=W_{\tt y}^0 \wedge O_4 < O_1) \vee	
%(R_{\tt y}^4=W_{\tt y}^1 \wedge O_1 < O_4 < O_3) \vee
%(R_{\tt y}^4=W_{\tt y}^3 \wedge O_3 < O_4)
%		\right) & \textbf{(thread interference)} \\
%\bigwedge & R_{\tt y}^4 > 2 \equiv {\bf true} & \textbf{(path conditions)} \\
%\bigwedge & t^2 \neq t^5 \wedge (2\ writes\ {\tt x} \vee 5\ writes\ {\tt x}) \wedge O_2 = O_5 & \textbf{(race condition)}
%	\end{array} 
%$$
%\end{center}
%\caption{\label{fig:encoding}\tool\ encoding of the trace in Figure \ref{fig:running} as a constraint system}
%\end{figure*}


We first preprocess the trace as follows. We introduce symbols to replace 
the shared accesses in the instructions, e.g., We use 
$W^{id}_{x}$/$R^{id}_{x}$ to replace the write/read of the shared 
variable $x$ by the 
event $e_{id}$. %The rationale is explained in Section~\ref{sec:relax1}.

{\bf Race Condition\ } A potential race pair involves two accesses of 
the same shared location from different threads, where at least one is write. 
Given any potential race pair, e.g., ($e_2$, $e_5$), it is a real race 
if and only if the two accesses can occur at the same time (race condition), 
which is captured by the race condition constraint: $O'(e_2) = O'(e_5)$.
The race condition constraint--- together with the other constraints --- 
guarantees the feasibility of the predicted race if a solution is found 
for the overall constraint system.

{\bf Intra-thread Constraints\ } Our analysis needs to preserve the 
same event sequence for each thread, which requires 

\begin{itemize}
\item {\bf Control Flow Constraints\ } Each thread takes the same control flow
(or branch decisions) to reproduce the events.
 For the running example (ignoring the statements in red), this 
yields: $\theta'(R_{\tt y}^4) > 2 \equiv {\bf true}$.
That is, in the predicted trace $\tau'$, the value of variable {\tt y} 
read at $e_4$ should be greater than $2$. 


\item {\bf Intra-thread Order Constraints\ } The events should follow 
the same thread-local order as in the original trace. This constraint 
is imposed by the fact that the two runs share the same instruction 
sequence and take the same control flow. 
We obtain the  formula: $O'(e_1) < O'(e_2) < O'(e_3) \wedge O'(e_4) < O'(e_5)$.
%That is, in the predicted trace $\tau'$, the three events from $T_1$ are 
%ordered in the same way as in $\tau$, and similarly, the branch 
%event $e_4$ in $T_2$ still executes before event $e_5$ inside the 
%branch body.

\item {\bf Intra-thread Value Constraints\ } The events should respect the 
constraints imposed by each instruction.
 For our running example, we obtain:
$$
{\small 
\begin{array}{l}
	\theta'(W_{\tt x}^0) = 0 \wedge \theta'(W_{\tt y}^0) = 0 \wedge \theta'(W_{\tt y}^1) = 3 \\ 
	\wedge	\theta'(W_{\tt x}^2) = 1 \wedge \theta'(W_{\tt y}^3) = 5
	\end{array}
}
$$
As an example, $\theta'(W_{\tt y}^3)= 5$ denotes that the predicted run still assigns the value 5 to the variable {\tt y}  at event $e_3$.
\end{itemize}

%The next set of constraints, denoted ${\tt w}^i=k$ for local variables and $W_{\tt w}^i=k$ for shared variables, capture variable definitions: Variable {\tt w} is assigned value $k$ at statement $i$.


%Indeed, this constraint is satisfied by the assignments to {\tt y} both at line {\tt 1} and at line {\tt 3}.


{\bf Inter-thread Value Constraints\ } These constraints capture what writes the read events read from and under what scheduling condition the value flow occurs.  The resulting formula for our example is
$$
{\small 
\begin{array}{lcl}
	& & (\theta'(R_{\tt y}^4)=\theta'(W_{\tt y}^0) \wedge O'(e_4) < O'(e_1) ) \\
& \bigvee &
	(\theta'(R_{\tt y}^4)=\theta'(W_{\tt y}^1) \wedge O'(e_1) < O'(e_4) < O'(e_3)) \\
& \bigvee &
	(\theta'(R_{\tt y}^4)=\theta'(W_{\tt y}^3) \wedge O'(e_3) < O'(e_4))
\end{array}
}
$$    
Notice, importantly, that the formula associates the value constraints 
with the scheduling order constraints. 
As an example, 
$\theta'(R_{\tt y}^4)=\theta'(W_{\tt y}^1) \wedge O'(e_1) < O'(e_4) < O'(e_3)$ 
means that, in the predicted run, the read $e_4$ reads from the write $e_1$, 
under the condition that $e_4$ happens after $e_1$ and no other  
writes (such as $e_3$) interleave them.

The formulas from the different encoding steps are conjoined and sent to an 
off-the-shelf solver such as {\sf z3}.  The race under analysis is real 
if the solver returns a solution, which includes the scheduling order 
and the values for the variables. 


\smallskip
\noindent
{\em II. Path Relaxation\ } 
%Next, we show how we catch the second missing race
%($e_2$, $e_7$) with another relaxation. 
Besides the first relaxation, which focuses on the given trace, the second
relaxation is to even reason about unexecuted branches as long as they
can be precisely modeled. In our 
running example, this is essential to discovering the race between 
lines {\tt 2} and {\tt 7}. We explore the unexecuted branch statically 
and collect constraints during the exploration. We then form a disjunction  
of the original constraints and the collected constraints to include more 
possible  paths. The solver may pick up the unexplored branch to detect 
a race. Back to our example, event $e_7$ is in the unexplored branch, \tool\ 
is able to identify the race between $e_7$ and $e_2$ after encoding 
the unexplored branch. 

For the racy events, $e_2$ and $e_7$, we first specify the race condition 
constraint  $\theta'(R^4_y)\leq 2\rightarrow O'_{e_2}=O'(e_7)$. Note that 
the constraint associates the race with the branch condition that 
enables each event.  Also we need to encode (1) the intra-thread constraints, 
such as the  order constraints $O'(e_4)<O'(e_5)\wedge O'(e_4)<O'(e_7)$, 
and (2) the inter-thread constraints, which remain the same as before. 
With these constraints, the solver confirms the race pair, ($e_2$, $e_7$). 
More details are in Section~\ref{sec:relax2}.



\smallskip
\noindent
{\em Highlights \ } \tool\  finds two more pairs of races 
beyond existing analyses~\cite{yannis, pldi14}: $(e_2, e_5)$ and $(e_2, e_7)$.
For the first race, we do not require $e_4$ to read from $e_3$ to 
retain the original value as in the
existing approaches. Rather, we allow $e_4$ to read from $e_1$ while 
still preserving the thread local path. This relaxation allows 
the events after $e_4$ to run concurrently with the events before 
$e_3$, leading to the detection of the first race.  For the second race, 
we cover unexecuted branches and query the solver to compute a feasible 
schedule that exercises them. 
%The neighboring branches are likely to 
%contain racy events if the executed branches contain racy events.

 
%In particular, the solution discloses a feasible trace that gives rise to the race at hand. The trace is identified uniquely via the order enforced in the solution over the variables $O_i$, which represent scheduling order. 
