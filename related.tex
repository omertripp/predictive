\section{Related Work}

%In this section, we survey related research on race detection with special emphasis on predictive trace analysis (PTA).

\subsection{Race Detection Techniques: Broad Survey} 

Initial attempts to address the challenge of race detection focused on 
built-in synchronization primitives \cite{eraser,SasturkarAWS05,vonPraun:2001,
Choi:2002}. These include locks as well as the wait/notify and start/join 
primitives. Notable among these efforts is the \emph{lockset} analysis, 
which considers only locks \cite{eraser}. Because the derived constraints 
are partial, permitting certain infeasible event reorderings, lockset 
analysis cannot guarantee soundness \cite{Naik:2006}. 

A different tradeoff is struck by the \emph{happens-before (HB)} 
approach \cite{Christiaens,Dinning:1990,Mellor-Crummey:1991}, which is 
founded on Lamport's HB relation \cite{Lamport}. In HB analysis, all 
synchronization primitives are accounted for, though reordering constraints 
are overly conservative. 
%As an example, HB inhibits reordering of two synchronized 
%blocks governed by the same lock. 

%PTA is an instance  of the HB approach, whose starting point is a concrete execution trace.

Recently there have been successful attempts to relax HB constraints, including
%Among these are 
the hybrid analysis \cite{hybrid}, which permits both orderings of 
lock-synchronized blocks, and the \emph{Universal Causal Graph (UCG)} 
representation~\cite{ucg}, which also enables both orderings but only 
if these are consistent with wait/notify- and start/join-induced constraints.

\subsection{Predictive Trace Analysis (PTA)}

Given a concurrent execution trace $t$, PTA derives new traces to witness 
the data races. PTA is founded on the notion of sound causality, as it 
considers feasible reorderings of the input trace that prove a candidate 
data race as such. 

Wang et al.~\cite{chao} pioneere the symbolic predictive analysis, 
which our approach also belongs to. However, the approach lacks support 
of heap encoding, therefore, cannot be applied to real-world applications. 
The first PTA that applies to real-world applications is proposed by 
Smagardakis et al. \cite{yannis}. In their original work, both synchronization 
constraints and inter-thread dependencies are preserved, where inter-thread 
dependencies are respected by only allowing reorderings that leave the 
dependence structure exhibited by the original trace intact. This ensures 
that the values of shared memory locations remain the same, which secures 
the soundness argument.

Said et al. \cite{Said:2011} and Huang et al. \cite{pldi14} proposed
PTAs that are also sound. Said et al. perform symbolic analysis of the 
input trace, and then utilize an SMT solver to search for schedules that 
establish the presence of data races. Soundness is guaranteed by their 
ability to precisely encode the semantics of sequential consistency. 
Inspired by Said et al., Huang et al. present an improvement, whereby 
control-flow information is encoded into the constraint system so as 
to relax flow dependencies as long as control dependencies are 
preserved. ExceptionNULL \cite{Farzan:2012} is another example of 
sound PTA. Its goal is to detect null dereferences rather than data races. 

\tool\ is similar to Said et al. and Huang et al. in that it also makes 
use of an SMT solver based on symbolic encoding of the input 
trace. Still, the two sound relaxations that \tool\ features, which 
enable (i) value- rather than dependence-based reasoning about data 
flow and (ii) consideration of unexplored branches, are strictly beyond
the scope of both techniques. In Section \ref{sec:eval}, we demonstrate 
via direct comparison with Huang et al. the substantial improvement 
in coverage thanks to these relaxation methods.
%, which we prove to handle 
%in a sound manner.

\subsection{Maximality Claim}
Serbanuta et al. \cite{maximal} and Huang et al.~\cite{pldi14} both 
claim maximality of the detection capability, but under different assumptions. 
Assuming branch events are not modeled in the trace, \cite{maximal} guarantees maximality. Given few 
information in the trace, it conservatively requires the shared read to 
obtain the same value in the predicted run to ensure soundness. However, if 
branch events are modeled, the technique does not guarantee maximality.  
Assuming the branch events are modeled but local assignments are not recorded, 
\cite{pldi14} guarantees maximality. 
Compared to \cite{maximal}, it allows some events to read different 
values if they do not affect the following branch decisions. However, 
it requires the events that affect the branch decisions to read exactly 
the same values. In practice, these assumptions are too restrictive  
because local assignments can be easily collected in practice.
%the testers often have full access to the source/bytecode and 
%can collect local assignments freely.  

Assuming local assignments are available in the trace, both techniques 
cannot guarantee maximality. As illustrated in our example and discussed 
in Section~\ref{sec:guarantee}, our analysis allows higher detection capability
%, i.e., we allow the events to read different values even they flow into the branch events; we allow the exploration of un-executed paths, both of which are not allowed by the above techniques.
 


