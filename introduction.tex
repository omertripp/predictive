%Limitations of static analysis: scalability when it is path sensitive, difficulty in modeling heap, cannot reason about concrete values, have to use
%abstraction, so that may not be sound, dynamic analysis is precise, easily scale to long and complex runs, but it only reasons about problems that
%did occur. 

%Predictive analysis is so good

%But it restricts too much, only relax schedule space and a little bit of the dependence space.

%We observe many dynamic aspects can be relaxed without affecting soundness.  as a result, instead of reasoning one
%execution, we can reason the neighborhood of the execution, in order words, we leverage dynamic analysis to provide
%the information that is difficult to analyze statically such as aliasing and a long execution path that covers the
%intended workflow, then we relax the predictive analysis to analyze neighborhood.  


\section{Introduction}\label{Se:introduction}
Program analyses can be largely classified to static and dynamic analyses. Static analyses abstract programs statically.
They are usually complete and do not require concrete program inputs to drive the analyses. However, they have difficulty in
scaling to large and complex programs when the analyses are path-sensitive, due to the sheer number of paths and the 
complexity of the paths. Path-insensitive analyses often do not have sufficient precision on the other hand. 
Static analyses also have difficulty in modeling complex aliasing behavior. In contrast, dynamic analysese focus 
on analyzing a few executions, usually just one. They are sound and easily scale to long and complex executions. But they 
can only reason about properties that manifest themselves in the execution(s). 

Recently, a novel kind of analysis called predictive analysis~\cite{...} was proposed. It has the capabilities of 
achieving soundness as dynamic analyses and reasoning about properties that did not occur during execution 
just like in static analysis. The basic idea is to leverage dynamic analysis to provide the information that 
is difficult to acquire through static analysis, such as a complex execution path that covers important functionalities 
and aliasing information. Note that during execution, we know precisely which write access reaches any given read 
access. While such information is encoded in a trace, predictive analysis reaons about mutation of the trace to 
expose problems or study properties that may not manifest during execution. It was used in sound data race detection. 
In particular, it collects a trace of threaded execution that contains rich runtime information such as concrete values, 
execution path, and memory accsses. It then leverages constraint solving to reason about if races can occur with
a different thread schedule, {\em while enforcing the critical runtime information (e.g. values, thread
local paths, and memory accesses) remains the same during schedule pertubation}. The rationale is that 
it becomes as difficult as static analysis if we allow these critical runtime information change. 
Preditive analysis has been shown to be very effective. It can detect real data races from complex programs.
And more importantly, it guarantees the results are sound (i.e. no false positives).

Despite its effectiveness, we observe that existing predictive analyes are too restrictive. They 
require too much runtime information to remain unchanged during analysis such that coverage is unnecessarily
limited. 


\begin{figure}
\centering
\begin{tabular}{ll}
\multicolumn{2}{c}{{\tt x = 0; y = 0;}} \\  % z = 3;
%\multicolumn{2}{c}{{\color{red} {\tt z = 2;}}} \\
\hline
\multicolumn{1}{c}{$T_1$} & \multicolumn{1}{c}{$T_2$} \\
\hline
{\tt 1: y = 3;} & \\
{\tt 2: x = 1;} & \\
{\tt 3: y = 5;} & \\
& {\tt 4: if (y > 2)} \\
& {\tt 5:~~z=1+x;} \\	
& {\color{red} {\tt 6: else}} \\
& {\color{red} {\tt 7:~~w=2+x;}}
\end{tabular}
\caption{Example illustrating ordering constraints beyond synchronization primitives}
\label{fig:running}
\end{figure}

Consider the example in Figure \ref{fig:running}, which shows an execution trace of two threads. The statements in red 
denote an unexplored branch outside the trace. Existing predictive analyses are unable to detect the race between 
lines {\tt 2} and {\tt 5} because they require that during trace permutations (by different schedules), a variable must not have
a different value in a permuted trace\footnote{They allow the variable involved in a race to have different values.}.
As such, they cannot execute line {\tt 4} before line {\tt 3}, which is needed to expose the race between lines {\tt 2}
and {\tt 5}. 
%the dependence between lines {\tt 3} and {\tt 4}, which is a barrier to the needed reorderings. 
A second race between lines {\tt 2} and {\tt 7} is also missed as the mutated trace must have the same thread local
path as the original trace. 

However, we observe that such restrictions are unnecessary. For example, we can allow line {\tt 4} to receive its
value from line {\tt 1} as it has the same effect on thread local execution (i.e. the same branch is taken). 
This relaxation would allow us to execute lines {\tt 4} and {\tt 5} before line {\tt 2}, thereby exposing 
the first missing race. We can also allow the trace to take the else branch such that the second race is also exposed. 
Note that even though that branch was not executed originally, its effect can be precisely modeled as the value 
of $x$ is known from the original trace. As such, we can still perform sound analysis.

%that is missed, as it involves the statements in red, is between lines {\tt 2} and {\tt 7}.
Therefore in this paper, we propose \sysname, a relaxed predictive analysis. We identify the critical runtime
informantion that needs to be preserved during trace perturbation and relax the remaining information. 
In particular, we preserve all the addresses de-referenced (e.g. heap and array accesses) in the original execution,
and some of the branch outcomes such that part of the thread local paths stay intact. The criterion is that
we forbid \sysname\ to explore an unexecuted branch if the effect of the branch cannot be precisely modeled
(e.g. an array element is read while it was not defined in the original trace). The essence of our technique 
is to {\em explore the neighborhood of the original execution as much as possible}. Due to the substantially
enlarged search space, our results show that a race detector based on our relaxed predictive analysis 
can detect XXX times more races than existing predictive analysis, while guaranteeing soundness. 

Our contrituions are summarized as follows.
\begin{itemize}
\item We propose to relax existing predictive analysis and identify a bound of relaxation (i.e. what can
be relaxed).
\item We develop a constraint encoding scheme that encodes the necessary information (i.e. the
part that must be preserved) and the relaxed information (i.e., the part that may be mutated) uniformly as part of a single logical representation.
\item We develop a propotype \sysname. Our results show XXX.
\end{itemize}








