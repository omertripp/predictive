\section{Constraint Derivation and Resolution}

We now describe in detail the constraint encoding and solving steps.

\subsection{Same Trace}

We begin with the basic setting, which enforces conservative reordering of the events along the input trace without diverging into new dependence structures or execution paths. We relax this requirement later, such that different data-flow paths and branches can be followed as long as feasibility is retained.

\paragraph{Intra-thread Order}

The first set of constraints reflects control flow within the individual threads, which must remain unchanged under the reordering transformation. Given input trace $t$, this is expressed as the following formula:
$$
\begin{array}{rl}
	\forall \tau,\tau' \in t. & {\sf thread}\ \tau \equiv {\sf thread}\ \tau'. \\
										& {\sf index}\ \tau < {\sf index}\ \tau' \Rightarrow O_{\tau} < O_{\tau'}
\end{array}
$$ 
The logical variables $O_x$ express ordering constraints. The requirement, as stated above, is that these variables
reflect the same order as the projection of ${\sf index}$ onto individual threads.

\paragraph{Race Condition}

Given pair $\tau$ and $\tau'$ of events that both access a common memory location $\ell$, we demand that
$$
\begin{array}{rl}
					& {\sf thread}\ \tau \neq {\sf thread}\ \tau' \\
\bigwedge 	& (\ell \in {\sf writeset}\ \tau \vee \ell \in {\sf writeset}\ \tau') \\
\bigwedge   & O_{\tau} = O_{\tau'}
\end{array}
$$
That is, (i) events $\tau$ and $\tau'$ are executed by different threads, (ii) at least one of the events performs write access to $\ell$ (as judged by the ${\sf writeset}$ membership check), and (iii) the events occur simultanesouly. This is a direct logical encoding of the definition of a race condition.

\paragraph{Path Constraints}

The requirement with respect to branching is that the prefix of the original trace $t$ up to the pair $\tau$ and $\tau'$ of candidate racing events remains identical in the predicted trace $t'$. More accurately, under the assumption that ${\sf index}\ \tau < {\sf index}\ \tau'$, 
we require that 
$$
	\bigwedge_{\tau'' \in t \cap {\bf bexp}.\
		\tau'' \in {\sf pre}\ \tau'} \lsyn {\sf stmt}\ \tau'' \rsyn\ t \equiv \lsyn {\sf stmt}\ \tau'' \rsyn\ t'
$$ 
where ${\sf stmt}$ is a helper function that obtains the code statement incident in a given transition. That is, all branching transitions up to $\tau'$ (which occurs after $\tau$) preserve their boolean interpretation under $t'$. This ensures that there are no divergences from the path containing the racing events, though beyond that path any feasible continuation is permitted. Importantly, contrary to \cite{JEFF-PLDI14}, we do not pose the requirement that the values flowing into branching statements remain the same, but suffice with the relaxed requirement that the evaluation of branching expressions is invariant under the input and predicted traces.

\paragraph{Variable Definitions}

A final requirement for the basic setting is that left-hand variables are defined according to the same right-hand variables as before. That is, version $i$ of variable ${\tt u}$ is defined as version $j$ of variable ${\tt v}$ in input trace $t$, then the same remains true in predicted  trace $t'$. This is enforced as the formula
$$
	\bigwedge_{\tau'' \in t \cap {\bf asgn}.\
		\tau'' \in {\sf pre}\ \tau'} {\sf stmt}\ \tau'' \in t' 
$$
where we again assume that ${\sf index}\ \tau < {\sf index}\ \tau'$. That is, the same statement occurring in $t$ is also present in $t'$ (though the transitions may differ). Since the statements of $t'$ are a permutation of the statements of $t$, we are assured that use/def flow is constrained appropriately.

\subsection{Relaxations}

We now move to the novel feature of \tool, which is its ability to explore execution schedules that depart from the data flow exhibited in the original trace. More precisely, \tool\ is able to relax flow dependencies in the original trace, whereby a thread reads a shared memory location written by another thread, while enforcing feasibility. This is strictly beyond the coverage potential of existing predictive analyses, which restrict trace transformations to ones where any read access to a shared memory location must correspond to the same write access as in the original trace.

\paragraph{Relaxation of Flow Dependencies}

To ensure feasibility under relaxation of flow dependencies, we need to secure the link between the execution schedule and the write/read flow. As an illustration from the example in Figure \ref{fig:running}, $R_{\tt y}^4={\tt y}^1 \wedge O_1 < O_4 < O_3$ specifies that in a schedule where thread $T_1$ executes line {\tt 1}, then $T_2$ executes line {\tt 4}, and then the schedule switches again to $T_1$ to execute line {\tt 3}, the read access to ${\tt y}$ at line {\tt 4} obtains the value assigned to ${\tt y}$ at line {\tt 1}: $R_{\tt y}^4={\tt y}^1$.

The full and general constraint formula, given read $R_{\ell}$ of location $\ell$ as part of event $e$ with set ${\cal W}$ of matching write events (i.e., events including write access to $\ell$), takes the following form:
$$
\begin{array}{rll}
\bigvee_{e_w \in {\cal W}} &  & (R_{\ell} = {\ell}^{{\sf index}\ e_w}) \\
&		\bigwedge 	&  O(e_w) < O(e) \\
&		\bigwedge_{e' \in {\cal W} \setminus \{ e_w \}} & (O(e') < O(e_w) \vee O(e) < O(e'))
\end{array}
$$
This disjunctive formula iterates over all matching write events, and demands for each that (i) it occurs prior to the read event ($O(e_w) < O(e)$) and (ii) all other write events either occur before ($O(e') < O(e_w)$) it or after the read event
($O(e) < O(e')$).

An important concern that arises due to relaxation of flow dependencies is that heap accesses may change their meaning. As an illustration, we refer to Figure \ref{fig:heapAccess}. While the read at line {\tt 7} appears to match the write at line {\tt 5}, this is conditioned on the read at line {\tt 6} being linked to the assignment at line {\tt 4}. If the predicted run violates this link, then feasibility is no longer guaranteed. In particular, if reordering results in {\tt z} being assigned the first rather than second allocated object, then the write at line {\tt 5} no longer matches the read at line {\tt 7}.
 
\begin{figure}
	\centering
	\begin{tabular}{ll}
		\hline
		\multicolumn{1}{c}{$T_1$} & \multicolumn{1}{c}{$T_2$} \\
		\hline
		{\tt 1: x1 = new();} & \\
		{\tt 2: x2 = new();} & \\
		{\tt 3: y = x1;} & \\
		{\tt 4: y = x2;} & \\
		{\tt 5: x2.f = 5;} & \\	
		& {\tt 6: z = y;} \\
		& {\tt 7: w = z.f;} \\
	\end{tabular}
	\caption{\label{fig:heapAccess}Example illustrating the need to account for heap accesses during trace transformation}
\end{figure}

To address this challenge, we enhance the constraint system with the requirement that heap objects that are dereferenced before the candidate racing events retain their original address in the predicted trace.
This achieves three guarantees: First, matching events in the original trace are guaranteed to also match in the predicted trace. Second, candidate races in the original trace remain viable in the predicted trace. Finally, in a practical setting involving virtual method calls (which goes beyond our core grammar in Table \ref{Ta:syntax}), call-site resolutions are the same across the original and predicted traces. Notice that in this setting, we consider as relevant not only the targets of field dereferences but also the targets of virtual method invocations.

Formally, given pair $\tau$ and $\tau'$ of candidate racing events such that ${\sf index}\ \tau < {\sf index}\ \tau'$, 
we require that
$$
\bigwedge_{\tau'' \equiv {\tt y=x.f} \in t \cap {\bf heapr}.\
	\tau'' \in {\sf pre}\ \tau'} {\sf env}\ \sigma(t,\tau'')\ {\tt x} = {\sf env}\ \sigma(t',\tau'')\ {\tt x}
$$
where ${\sf env}$ is the state mapping from local variables to their value and $\sigma(t,\tau'')$ 
(resp. $\sigma(t',\tau'')$) is the state arising at trace $t$ (resp. $t'$) immediately before event $\tau''$.
This constraint fixes that all heap dereferences up to the later of the candidate racing events retain their original base object as in the predicted trace $t'$. 

\paragraph{Relaxation of Branching Decisions}

Beyond relaxing flow dependencies, \tool\ is also capable of exploring code branches that were not executed in the original trace. This is achieved via a symbolic representation of the input trace. Given branching event $e_b$,
\begin{enumerate}
	\item model the branching condition symbolically to execute along the negation of the original branch;
	\item apply depth-first search (DFS) to uncover all execution suffixes under the unexplored branch (which may itself contain branching statements); and
	\item for each suffix $t^s$,
	\begin{enumerate}
		\item truncate the original trace at $e_b$ yielding prefix $t_s$; and
		\item concatenate $t_s$ with the negation of $e_b$ followed by $t^s$.
	\end{enumerate}
\end{enumerate}
Constraint solving is applied to each of the resulting traces analogously to the original trace.

We emphasize that exploration of new execution paths is subject to all the known limitations of symbolic execution, including in particular loop structures and object allocation. \tool\ currently fails if (i) the unexplored branch containts loops or (ii) there are object references that cannot be fully resolved at the branching point. In case of failure, \tool\ moves on to other branches. We demonstrate in Section \ref{sec:eval} that despite these limitations, the increase in coverage thanks to exploration of new paths is significant.

\subsection{Proof of Correctness}

Researchers~\cite{pldi14,maximal} propose two axioms, {\em prefix closedness} and {\em local determinism}, which we adapt to establish our feasibility guarantee. 
Suppose $\mathcal{F}$ denotes the domain of feasible traces.


Intuitively, Prefix closedness means that if a trace is feasible, then the trace is still feasible after we discard the events after a point. The underlying reason is that the events after a point cannot affect the events before it. More formally,

\begin{myaxiom}[Prefix closedness]~\label{axiom:prefix}
If $t_1$ $t_2$ $\in \mathcal{F}$, then $t_1 \in \mathcal{F}$. 
\end{myaxiom}




Intuitively, local determinism says that only the previous events of the same thread determine the existence of an event. For example, the existence of an event is only determined by the branch  thread-locally. However, the value read by an event may be affected by another thread because a shared read reads from the most recent write, which may come from a different thread. The value in other events, such as branch or write,  are locally determined too.
More formally,





\begin{myaxiom}[Local determinism]~\label{axiom:local}
Assume $t_1e_1, t_2 \in \mathcal{F}$, and $t_1|thread(e_1)\approx t_2|thread(e_1)$, where $\approx$ denotes the two traces are equal if the data values in the read and write events are ignored, 
Each event is determined by the previous events in the same thread. There are four cases:
\begin{itemize}
%TODO refine it.
\item {\bf Branch\ }. If op($e_1$)=branch, and $e_1$ is in the form of $x<y$ (without loss of generality), then there exists values $v_x$, $v_y$ such that $t_2 e_1 [v_x/data_x, v_y/data_y] \in \mathcal{F}$. Here $e_1 [v_x/data_x]$ represents that we replace the value of $x$ from $data_x$ to $v_x$ in $e_1$.
\item {\bf Read\ }. If op($e_1$)=read, and $e_2$ is a read event that is identical to $e_1$ except that it may read a different value, and $t_2 e_2$ is consistent, then $t_2 e_2 \in \mathcal{F}$.
\item {\bf Write\ }. If op($e_1$)=write, and there is a value $v$ such that $t_2 e_1[v/data]\in \mathcal{F}$. 
\item {\bf Others \ }. If op($e_1$) is of type different from above and $t_2 e_1$ is consistent, then $t_2 e_1 \in \mathcal{F}$.
\end{itemize}

Here, the consistency is defined as follows. 
A trace is consistency iff it satisfies (1) read-write consistency, i.e., read event should contain the value written by the most recent write event, and (2) the synchronization consistency, e.g., the lock acquire and release of a lock should not be interleaved by the lock operations of the same lock, the begin event of a thread should follow the fork event of the parent thread.
\end{myaxiom}

The local determinism resembles the version in ~\cite{pldi14,maximal}. A crucial difference is that, for the branch event, we do not require all reads thread locally before it to read the identical values as in the original trace $t_1 e_1$. Instead, we allow the previous reads to read different values as long as the branch event is evaluated to the same boolean value, which guarantees the existence of the following events. Such relaxation allows us to find more races, while still preserving the feasibility. 


An important assumption of ours is, we assume all local accesses are recorded in the trace so that we can determine the boolean status of a branch event computationally. ~\cite{pldi14}, which assume the local accesses are not recorded and cannot determine the status computationally, have to enforce the reads before a branch to read the same values, leading to a weaker feasibility criterion. We believe our axiom captures the real-world scenario more faithfully.
Another important assumption is for the read-write consistency, i.e., we assume the heap invariant, so that we know exactly which reads correspond to which writes in the predicted run.

%\begin{myaxiom}
%A set of traces $\mathcal{F}$ is feasible if it satisfies prefix closedness and local determinism axioms. 
%\end{myaxiom}



They are called axioms because they are the rules that the sequentially consistent system~\cite{maximal} should follow.
We refer the readers to the detailed discussion~\cite{maximal, pldi14}.



%TODO what is \Phi
\begin{mytheorem}[Soundness]
All traces in the closure $closure(t, \Phi)$, which includes $t$ and is closed under the derivations in Axioms~\ref{axiom:prefix}~\ref{axiom:local}, are feasible.
\end{mytheorem}

%$\theta_1(expr(e_1))=\theta_2(expr_{e_1})$, then $t_2 \e_1 \in \mathcal{F}$. Here $\theta_1, \theta_2$ represent the mappings between variables and values established prior to $e_1$ by the same thread $thread(e_1)$. $expr(e_1)$ is a helper function that returns the expression in the branch condition. 
\begin{proof}
We order the traces in $closure(t)$ as, $t_0=t, t_1, t_2, t_3, \dots t_n$. Clearly, $t_0$ is feasible. In the following, we prove by induction, i.e., assume $t_{n}$ is feasible, then $t_{n+1}$ which is derived from $S=\{t_0, \dots t_n\}$ should be feasible. There are several possible derivations.
\begin{itemize}
\item if $t_{n+1}$ is a prefix of $t'\in S$, then $t_{n+1}\in\mathcal{F}$ because of the prefix closedness. $t_{n+1}$ shares the same mappings as $t'$.
\item if $t_{n+1}$ is derived from $t^1 e^1, t^2\in \mathcal{F}$, we have $t^1|thread(e^1)\approx t^2|thread(e^1)$.
\begin{itemize}
\item $op(e^1)=branch$ and $e^1= x<y$. Our solver computes a mapping $\theta^d$ of all relevant variables such that $\theta(x<y)=\theta^d(x<y)$, then 
$t^2 e_1[\theta^d(x)/\theta(x),\theta^d(y)/\theta(y) ]  \in \mathcal{F}$. Note that the mapping computed by solver may be different from the mapping in the trace $t^1$.
\item $op(e^1)=read$. Our solver ensures the read-write consistency, i.e., $t^2 e^2$ is consistent, then $t^2 e^2 \in \mathcal{F}$. Note that $e^2$ is the same as $e^1$ except that it may read a different value.
\item $op(e^1)=write$. There exists a value $v$ such that $t^2 e^2[v/data]\in \mathcal{F}$, where $v$ is included in the mapping computed by the solver. 
\item $op(e^1)$ is of other types. Our solver ensures the consistency, i.e., $t^2 e^1$ is consistent, then $t^2 e^2 \in \mathcal{F}$.
\end{itemize}
\end{itemize}
\end{proof}



\begin{mytheorem}[Maximality]
$\forall t', s.t., t'|th\approx t|th for any thread th, if t'\notin closure(t,\Phi), t' is infeasible.$
\end{mytheorem}

\begin{proof}
We sketch the proof. All traces that satisfy  $t'|th\approx t|th for any thread th$ 
\end{proof}



%discussion.