\section{Relaxation of Flow Dependencies}~\label{sec:relax1}
Given the trace, $\tau=<\Gamma , \{\tau_{t_1}, \tau_{t_2}, \dots \tau_{t_n} \}, O, \theta>$, the main goal of our analysis is to derive a new trace,  $\tau'=<\Gamma , \{\tau_{t_1}, \tau_{t_2}, \dots \tau_{t_n} \}, O', \theta'>$, which reschedules the events from different threads (the order of events inside each thread should remain unchanged) to witness the race. The key insight underlying our first relaxation is that, the read events in $\tau'$ are allowed by our analysis to read different values from different writes (than in trace $\tau$), but they are required by existing approaches to read the same values as in $\tau$. Therefore, our relaxation allows more schedules, which are likely to expose more races. Note that we also need to compute the new value mapping $\theta'$ because the rescheduling may alter the value every variable is assigned to. The computation of the new schedule and new value mapping is explained in the following.






%We begin by describing the first relaxation, whereby the dependence structure defined by the original trace is potentially violated while retaining soundness.

\subsection{SSA Form of the Trace}~\label{sec:ssa}
First of all, we rewrite the trace into SSA form, such that every variable in defined exactly once in the trace. This requirement is a prerequisite for the computation of the mapping, in which each variable is mapped to exactly one value. Another side effect is that the def/use chains become explicit in the SSA form, which simplifies the following analysis steps.

\tool\ handles the local assignment and heap accesses differently.
\begin{itemize}
\item {\bf Local Assignment\ } The SSA form for local assignment resembles the SSA form of static instructions in compiler optimization, except that the loops and recursions are fully resolved in a concrete trace, obviating the need for the {\sf Phi} node. More concretely, we replace the variable $v$ defined in an event, as well as the following uses of the definition, to a new variable $v^{id}$, where $id$ is the unique id of the event. The uniqueness of the id guarantees that no two events define the same variable, i.e., each variable is defined exactly once.  Consider the example in Figure~\ref{fig:running2}, where the line number is also used as event id,  the local variable $x$ defined at line 4 is renamed as $x^4$ in the trace, while $x$ defined at line 9 is renamed as $x^9$, which is used in the following two events.  At line 5, where the object $o3$ is created,  we define a reference variable $w^5$.  Internally,  the object $o3$ is encoded as a unique integer that indicates its heap address.
\item {\bf Heap accesses\ } The def/use relation between local heap accesses can be locally determined. Therefore, we treat them in the same way as local assignment. The def/use relations between shared heap accesses are more complex. We may link a read with different writes under different schedules, e.g., the read access to {\tt y} at line 4 in Figure \ref{fig:running} can either obtain the definition before line 1 or at line 1 or at line 3. Therefore, we introduce two symbols to represent the shared read and shared write respectively and use them to establish different links flexibly. The details are as follows. 
\begin{itemize}
\item {\bf Local Heap Accesses\ } Given the write $x.f=y$ to a local heap location at event $e_{id}$,  we introduce a fresh local variable {\tt $l^{id}_{o.f}$} ($o$ is the runtime object referenced by $x$) to replace $x.f$. We also replace the uses of the location $o.f$ accordingly.
Here  the subscript $o.f$ is interpreted as a heap location, i.e., the field $f$ of the object $o$. Consider the example in Figure~\ref{fig:running2}, at lines 6-7, we replace the local heap accesses as the local variable $l^6_{o3.f}$.
%An assumption here is the uses and the definition still refer to the same base object $o$ in the predicted run, which is enforced through the constraint specification %(Section~\ref{sec:relax1}).
\item  {\bf Shared Heap Accesses\ }    Given the write $x.f=y$ or the read $y=x.f$ of a shared location at event $e_{id}$, we introduce two symbols, $W^{id}_{\tt o.f }$ and $R^{id}_{\tt  o.f }$ respectively, where $o$ is the runtime object referenced by the $x$. Similarly, the subscript $o.f$ is interpreted as a heap location, i.e., the field $f$ of the object $o$.
\end{itemize}
We use the heap location such as $o.f$ in the the symbolic form of heap accesses. An underlying assumption is that the heap location remains unchanged in our predictive analysis, which we ensure through additional constraints (Section~\ref{sec:relax1}).
\end{itemize}


% Similar to the local heap accesses, an assumption for the correlation is that the writes and reads still refer to the same base object $o$ in the predicted run, which we guarantee by specifying additional constraints (Section~\ref{sec:relax1}). 

% We could encode the question of shared versus local accesses as additional constraints (requiring that a variable be accessed by more than one thread), but that would be more complicated and less efficient.


%\paragraph{Basic Encoding: Local Accesses}
%
%The fundamental encoding transformation is to induce Static Single Assignment (SSA) form on the raw trace, such that
%a variable is defined exactly once. In this way, , and encoding of trace events as constraints is simplified. As an illustration, trace
%\begin{quote}
%	{\tt 1: x=1; 2: x<3; 3: x=3;} \\
%	{\tt 4: y=1;} \\
%	{\tt 5: z=x+y}
%\end{quote}
%becomes
%\begin{quote}
%	{\tt 1: x$^1$=1; 2: x$^1$<3; 3: x$^3$=3;} \\
%	{\tt 4: y$^4$=1;} \\
%	{\tt 5: z$^5$=x$^3$+y$^4$}
%\end{quote}
%(For readability, we version variables according to the line number of their definition.)



%%TODO update the intro+moti, make sure the same style
\begin{figure}
\centering
\begin{tabular}{ll|l}
\multicolumn{3}{c}{{\tt {\bf s1} = new();//$o1$}} \\
\multicolumn{3}{c}{{\tt {\bf s2} = new();//$o2$}} \\
\multicolumn{1}{c}{$T_1$} & \multicolumn{1}{c}{$T_2$}  &  \multicolumn{1}{c}{$Trace$}\\
{\tt 1: {\bf s1}.f=3; } &  & {\tt $W^{1}_{o1.f}$=3;}\\
{\tt 2: {\bf s2}.f=1; } &  & {\tt $W^{2}_{o2.f}$=1;}\\
{\tt 3: {\bf s1}.f=2; } & & {\tt $W^{3}_{o1.f}$=2;} \\
& {\tt 4: x=0;} & {\tt $x^4$=0;}\\
& {\tt 5: w=new();//$o3$} & {\tt $w^5$=$o3$;}\\
& {\tt 6: w.f=10;}  & {\tt $l^6_{o3.f}$=10;}\\
& {\tt 7: z=w.f;} & {\tt $z^7$=$l^6_{o3.f}$;}\\
& {\tt 8: y={\bf s1}.f;} & {\tt $y^8$=$R^8_{o1.f}$;}\\
& {\tt 9: x=y+z;} & {\tt $x^9$=$y^8$+$z^7$;}\\
& {\tt 10: if(x>11)}& {\tt if($x^9$>11)}\\ 
& {\tt 11: \ \ {\bf s2}.f=x;}  & {\tt $W^{11}_{o2.f}$=$x^9$;}\\
\end{tabular}
\caption{Running Example (shared variables are in bold font, the line number is used as the event id). }
\label{fig:running2}
\end{figure}

 
%\begin{figure}
%\centering
%\begin{tabular}{l|l}
%\hline
%\multicolumn{1}{c}{$Trace$} & \multicolumn{1}{c}{$SSA \ form \ of\   Trace$} \\
%\hline
%{\tt 0: {\bf x}=0} &  {\tt 0: $W^0_x$=0}    \\
%{\tt 1: {\bf y}=0} &   {\tt 1: $W^1_y$=0}   \\
%{\tt 2: s=0} &  {\tt 2: $s^2$=0}   \\
%{\tt 3: i=1} &     {\tt 3: $i^3$=1}   \\
%{\tt 4: i<2} &    {\tt 4: $i^3$<3} \\
%{\tt 5: s=s+i} & {\tt 5: $s^5$=$s^2$+$i^3$}   \\
%{\tt 6: i=2} &       {\tt 6: $i^6$=2}  \\
%{\tt 7: i<2} &      {\tt 7: $i^6$<3}  \\
%{\tt 8: {\bf y} = s;} &  {\tt 8: $W^{8}_y$ = $s^5$;}  \\
%{\tt 9: {\bf y} > 2}  &    {\tt 9: $R^{9}_y$ > 2} \\
%{\tt 10: print({\bf x});} &  {\tt 10: print($R^{10}_x$);}  \\
%\end{tabular}
%\caption{Trace}
%\label{fig:t4running2}
%\end{figure}

%Figure~\ref{fig:t4running2} exemplifies a trace and its SSA form, where the trace is generated from the program in Figure~\ref{fig:running2}.
%Each label in  Figure~\ref{fig:t4running2} on the left hand denotes the id of the event. As seen, event $e_5$ uses the variable $s^2$ defined at event $e_2$ and defines the variable $s^5$, which is used at $e_8$. Reads and writes of shared variables are denoted in the special form explained above (e.g.: $W^{8}_y$ and  $R^{9}_y$).





\subsection{Constraint System}~\label{sec:constraints}
Based on the SSA form of the original trace $\tau$, we build the constraints to compute a new trace $\tau'$ with the new schedule $O'$ and new value mapping $\theta'$.
Consider the example in Figure~\ref{fig:running2}, in the original trace $\tau$, the event $e_8$ reads the value of 2 from $e_3$, therefore, $e_{11}$ must happen after $e_2$. Existing approaches enforce the same value dependence between $e_3$ and $e_8$, leading to the miss of the race between $e_2$ and $e_{11}$.
Our analysis can derive a new trace $\tau'$ to find the race by relaxing the dependence. In $\tau'$, $e_8$ reads from a different event $e_1$, producing the new value mapping, $\theta'(y^8)=3$. Accordingly, at the event $e_9$, the value mapping for $x^9$ is also updated as 13 ($\theta'(y^8)$=3, $\theta'(z^7)=10$), enabling the true branch at line 10. Finally, the events $e_2$ and $e_{11}$ are scheduled to run concurrently.  The relaxation needs to respect a set of constraints, which we explain in the following. We omit the synchronization constraints intentionally, as they are well explained in all existing predictive analysis techniques~\cite{yannis, pldi14}.


%Having explained how the trace is encoded, we now describe in detail how constraints are derived from the trace, such that any permutation considered by the analysis is guaranteed to represent a feasible execution schedule.

{\bf Race Condition\ } Following the standard definition, pair ($e_i$, $e_j$) of events forms a race iff (1) $e_i$ and $e_j$ are accesses of the same location $\ell$ by different threads, (2) at least one of them writes to $\ell$, and (3) $e_i$ and $e_j$ run concurrently. We refer to the candidate pair throughout this section as (candidate) \emph{racy events}.  

We first identify all candidate pairs of events that access the same location from different threads (and at least one is a write). We then check each pair  separately.   The checking is achieved by encoding all necessary constraints and invoking a constraint solver.  The first constraint asserts the feasibility of the concurrent execution of the racy events:
$$
O'(e_i) = O'(e_{j})
$$
In Figure~\ref{fig:running2}, the race condition for the race pair, ($e_2$, $e_{11}$), is $O'(e_2)=O'(e_{11})$


{\bf Intra-thread Constraints\ } Given the candidate race pair, $e_i$ and $e_j$, suppose $e_i$ is after $e_j$ in the original trace $\tau$. The predictive analysis reschedules the events in the prefix of $e_i$, i.e., prior to $e_i$, so that $e_i$ and $e_j$ can run concurrently. We refer to the prefix as $prefix(\tau, e_i)$. Throughout this section, we only consider the events in the prefix. 

The predictive analysis by design requires that each thread should follow the same event sequence prior to $e$ as in the original run.  The preservation of the event sequence for each thread requires the following intra-thread constraints:

\begin{itemize}
\item {\bf Control Flow Constraints\ } The branches in the predicted run should take the same decisions as in the original trace so that each thread reproduces the same set of events.  Specifically, we only need to reason about the branches prior to $e_i$.

Without loss of generality, we assume the branch event $e_k$ is in the form of $if(x<y)$. Then  we require that 
$$
\bigwedge_{e_k \in prefix(\tau,e_i) \wedge  inst^{e_k}=if(x<y).\ }  \theta'(x)<\theta'(y) 	\equiv \theta(x)<\theta(y)
$$ 
The constraint specifies that the branch condition should be evaluated as the same boolean value in the predicted run $\tau'$ as in the original trace $\tau$.
 Importantly, Unlike existing analyses \cite{yannis,pldi14}, we do not pose the requirement that the values flowing into branching statements remain the same, but adopt the relaxed requirement that the evaluation of branching expressions remains the same. Consider the example in Figure~\ref{fig:running2}, given the branch if($x^9$>11) at $e_10$, the original mapping is $\theta(x^9)=12$. Existing analyses require the same value mapping for $x^9$  in the predicted run, while we allow  a different value mapping, e.g., $\theta'(x^9)=13$, which retains the same truth value for the branch expression.
\item {\bf Intra-thread Order Constraints\ } The events in the sequence should follow the same order as in the original trace. More formally, we require that
$$
\begin{array}{rl}
\forall e_m, e_n \in prefix(\tau,e_i), s.t., & t^{e_m} = t^{e_n}. \\
 O(e_m) < O(e_n)\  & \Rightarrow O'(e_{m}) < O'(e_{n}) 
\end{array}
$$ 
This constraint specifies that two events from the same thread should follow the same order as reflected by the ids of the events.
\item {\bf Intra-thread Value Constraints\ } The value mapping of the variables should not contradict the value constraints imposed by each instruction.   Without loss of generality, suppose the instruction is a local assignment in the form of $x=y+z$, then we require that
$$
\bigwedge_{e_k \in prefix(\tau,e_i) \wedge inst^{e_k}=x=y+z}\
	\theta'(x)=\theta'(y)+\theta'(z)
$$
Remember that every variable may obtain a different value in the new trace $\tau'$ because it may read from some shared variable affected by the rescheduling.
The constraint specifies that, the variables should be re-assigned to some values that are consistent with the instruction. Consider the example in Figure~\ref{fig:running2}, given the event $x^9=y^8+z^7$ at line 9, if the mappings $\theta'(y^8)$ and $\theta'(z^7)$ change to 3 and 10 respectively, the mapping $\theta'(x^9)$ should change to 13 consistently. In the above, we use the local assignment event for demonstration, in general, the intra-thread value constraints should also be specified for the {\sf  heapr/heapw} events.
\end{itemize}



%Rescheduling through Relaxation of Flow Dependencies
{\bf Inter-thread Value Constraints for Relaxation\ } We now move to the first novel feature of \tool, which is its ability to explore execution schedules that depart from the value flow exhibited in the original trace. More precisely, \tool\ is able to relax value flow dependencies in the original trace:  a read access may read a
different value from other write events, as long as the read value enforces feasibility of the following execution.
 This is strictly beyond the coverage potential of existing predictive analyses, which restrict trace transformations to ones where any read access must read the same  value (often from the same write event) as in the original trace. 

To ensure feasibility under relaxation of flow dependencies, we need to secure the flow between the read/write with the execution schedule. 
For example, in Figure \ref{fig:running2}, for the read at $e_8$ to read from the write t $e_1$, we need a schedule where 
$e_8$ happens after $e_1$ and other writes such as $e_3$ do not interleave them.


% $R_{\tt y}^4={\tt y}^1 \wedge O_1 < O_4 < O_3$ specifies that in a schedule , the $R_{\tt y}^4={\tt y}^1$.


In general, the constraint formula, given read $R^{m}_{\ell}$ of location $\ell$ at the event $e_m$ with set ${\cal W}$ of matching write events (i.e., events including write access to $\ell$), takes the following form:
$$
\begin{array}{rll}
\bigvee_{e_n \in {\cal W}} &  & (\theta'(R^{m}_{\ell}) = \theta'(W^{n}_{\ell})) \\
&		\bigwedge 	&  O'(e_n) < O'(e_m) \\
&		\bigwedge_{e_p \in {\cal W} \setminus \{ e_n \}} & (O'(e_p) < O'(e_n) \vee O'(e_m) < O'(e_p))
\end{array}
$$
This disjunctive formula iterates over all matching write events, and demands for each that (i) it occurs prior to the read event $O'(e_n) < O'(e_m)$ and (ii) all other write events either occur before $O'(e_p) < O'(e_n)$ it or after the read event
$O'(e_m) < O'(e_p)$.

An important concern that arises due to relaxation of flow dependencies is that heap accesses may change their meaning, i.e., they involve different base objects and no longer match with each other. As an illustration, we refer to Figure \ref{fig:heapAccess}. While the read at the event $e_7$ appears to match the write at $e_5$, this is conditioned on the read at $e_6$ being linked to the assignment at $e_4$. Suppose  the event $e_6$ reads from $e_3$ due to the relaxation. Then $e_7$ and $e_5$ no longer share the same base object (or the same location). Even worse, we do not know which events $e_7$ matches, because the base object is no longer known.



\begin{figure}
	\centering
	\begin{tabular}{ll}
		\hline
		\multicolumn{1}{c}{$T_1$} & \multicolumn{1}{c}{$T_2$} \\
		\hline
		{\tt 1: $x^1$ = new();// creates o1} & \\
		{\tt 2: $x^2$= new();// creates o2} & \\
		{\tt 3: $W^3_{y}$ = $x^1$;} & \\
		{\tt 4: $W^4_{y}$ = $x^2$;} & \\
		{\tt 5: $W^5_{o2.f}$ = 5;} & \\	
		& {\tt 6: $z^6$ = $R^6_{y}$;} \\
		& {\tt 7: $w^7$ = $R^7_{o2.f}$;} \\
	\end{tabular}
	\caption{\label{fig:heapAccess} A trace.}
\end{figure}

{\bf Invariant Base Object Constraints\ } To address this challenge, we enhance the constraint system with the requirement that heap objects that are dereferenced in field access statements before the racy events retain their original address in the predicted trace.
This achieves two guarantees: First, matching heap access events in the original trace are guaranteed to also match in the predicted trace. Second, candidate races in the original trace remain viable in the predicted trace as they still refer to same location. Third, sharing between threads of heap locations remains unchanged  because
each location is accessed by the same number of threads given that the base object is the same. 





%Note that, the resolution of the virtual method call is modeled as the branch events that check the type of the target object, for which the constraints naturally guarantee the branch to take the same decision, i.e., the call to be resolved to the same method implementation.

%Finally, in a practical setting involving virtual method calls, call-site resolutions are the same across the original and predicted traces. 
%Notice that in this setting, we consider as relevant not only the targets of field dereferences but also the targets of virtual method invocations.

Formally, we require that
$$
\bigwedge_{ e_k \ has\  x.f \wedge e_i \in prefix(\tau,e_i) }\
	\theta'(x) \equiv  \theta(x)
$$
%where ${\sf env}$ is the state mapping from local variables to their value and $\sigma(t,\tau'')$ 
%(resp. $\sigma(t',\tau'')$) is the state arising at trace $t$ (resp. $t'$) immediately before event $\tau''$.


This constraint fixes that all heap dereferences  prior to the racy event retain their original base object as in the original trace $\tau$. 
The array accesses are handled similarly to the field accesses, but we need to additionally require the index variable to be the same as in the original trace.
In addition, for local heap accesses, in case that the base variable reads a local object from a shared reference variable, the base variable may change due to the rescheduling. We also fix the base variable as a constant, in order to ensure the validity of the SSA encoding of the local heap accesses and preserve the def/use chains among them. The constraint is specified during the SSA encoding.




By sending the above constraints to a solver, we compute the necessary schedule orders among the events as well as the mapping of the variables. The necessary schedule orders define a partial order among the events, which permit a set of schedules that define the complete order that complies with it.


