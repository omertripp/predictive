\begin{abstract}
Predictive analysis has recently emerged as a promising technique for 
sound detection of multithreading bugs (i.e., without false alarms). 
The idea is to start from an instrumented execution trace, and transform 
it based on the concrete information it provides, such as concrete values 
and memory accesses, to expose bugs, while preserving feasibility. For
example, it may reoder trace events to simulate different thread schedules, 
as long as flow dependencies are preserved.
Predictive analysis has been successfully applied to data-race detection, 
although the state-of-the-art techniques are overly restrictive in the 
transformations they allow such that their analysis coverage is 
unnecessarily limited. 
%so as to guarantee feasibility 
%and consequently also soundness. In particular, existing techniques (i) do not permit flow dependencies to be violated (except in very specific cases), and (ii) are unable to analyze execution paths beyond that of the input trace.
In this paper, we present \sysname, a predictive race detector that has two
important relaxations: value and path relaxations without affecting the
soundness guarantee. The former allows 
variables to have different values and the latter allows exploring 
unexecuted paths that are neighbors to the traced path.
% both of these restrictions while guaranteeing soundness, as we prove formally. Thanks to its ability to explore unexecuted branches as well as event orderings that violate flow dependencies, 
Our results show that \sysname\ is able to detect x$2.5$ more data races on a 
representative set of large-scale benchmarks compared to the state-of-the-art.
%in predictive race detection.
\end{abstract}
